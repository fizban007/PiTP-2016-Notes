\documentclass[letterpaper, 11pt]{article}

\usepackage[pdftex]{graphicx}
\usepackage{epstopdf}
\DeclareGraphicsRule{*}{mps}{*}{} 

\usepackage{amsmath, amsthm, amssymb}
\usepackage{listings}
\usepackage{float}
\usepackage{enumerate}
% \usepackage{mystyle}
\usepackage{hyperref}
\usepackage{tikz}
\usepackage{fancyheadings}
\usepackage{tensor}
\usepackage{mathrsfs}
\usetikzlibrary{positioning}
\usetikzlibrary{decorations.pathmorphing}
\usetikzlibrary{arrows}
\usetikzlibrary{decorations.markings}
%\usepackage{fullpage}
\usepackage[left=0.75in, top=1.25in, right=0.75in, bottom=1.25in]{geometry}
\newcommand{\lambdabar}{{\mkern0.75mu\mathchar '26\mkern -9.75mu\lambda}}
\newcommand{\Jmath}{J}

\numberwithin{equation}{section}
\numberwithin{figure}{section}

\begin{document}

\title{Black Hole Magnetospheres}
\author{Alexander Tchekhovskoy}
\date{July 21, 2016}

\maketitle

\section{Lecture 1}

Lets start with an overview of where do we see black holes in the universe.
There are supermassive black holes with $M\sim 10^{6-10}M_{\odot}$, which sit in
the center of galaxies. There are also stellar mass black holes we can see,
which could be in binaries, or associated with Gamma-ray bursts or mergers with
neutron stars. Recently there is also evidence of $M\sim 10^{2-5}M_{\odot}$
intermediate mass black holes which might be ultra-luminous X-ray sources. All
of these black holes produces jets. We don't really need an event horizon to
produce a jet though. We can have jets from stars, neutron stars and WDs as
well. There must be some universal properties of jet production that do not
depend too much on the exact dynamics of the central object.

How do we observe jets? There is a famous jet 3C279 which is traveling towards
us at $v\approx 0.997c$. Its sky location appears to be moving superluminally,
but it is an artifact of it moving towards us. We observe these jets at multiple
wave bands, from radio, IR/optical, to X-rays and Gamma rays. We can also
potentially detect cosmic rays, neutrinos, and gravitational waves from them.

Jets can also affect the galaxies and clusters around them. Examples can be the
Perseus cluster, M87, and MS0735.6 systems. The jets extract a large fraction of
the black hole energy and throw them into the ambient environment.

The masses of the central BHs and their host galaxies rotation curve are
correlated. This indicates a way for the BH to know about the galaxy around it.
The AGN radio luminosity also shows radio quiet and radio loud dichonomy, which
seems to suggest that there is at least an additional hidden parameter about the
jet other than $M$ and $\dot{M}$. Possibilities can be magnetic flux difference,
effect of the ambient medium instead of the BH itself, or it could be BH spin.

Different jets can also have different morphology, for example the Cygnus A
galaxy and M87 galaxy, which host similar mass black holes, but have very
different image of their jets (Fanaroff \& Riley 1974).

In April there will be more observing tools available in the form of the Event
Horizon Telescope (EHT), which can directly look at the vincinity of the black
hole in the multiple of horizon size.

Lets look at a simple illustration of how jet is produced. Consider a rotating
central object and a ceiling above it which is perfectly conducting,
representing the ambient medium. Starting with a straight field line, since it
is frozen into the ambient medium, as the central object rotates the magnetic
field will be wound up, and when there is enough pressure due to the development of
magnetic pressure it will push away and into the interstellar medium.

What powers the outflow? Rotation plus large scale magnetic flux leads to jets.
In reality, the black hole is very simple. It has only two hairs: mass and spin.
It knows nothing adbout the magneic flux. However, external accretion can
provide magnetic flux threading the black hole event horizon.

\dots

\subsection{Jet Acceleration}

Consider a neutron star which is not rotating, with field lines sticking out
from the surface. If we set it rotating, the field lines will start to lag
behind. The field line perturbation will travel along the field at Alfven speed.
We assume the jets are massless (force free) for simplicity. In this limit the
Alfven velocity will be speed of light $v_{A} = c$. We can find $B_{\phi}$ by
\begin{equation}
  \label{eq:1}
  B_{\phi} = -\frac{v_{\phi}}{c}B_{r} = -\frac{\Omega R}{c}B_{r}
\end{equation}
The electric field is exactly the same as the toroidal magnetic field
\begin{equation}
  \label{eq:2}
  E = \left| -\frac{\mathbf{v}}{c}\times \mathbf{B} \right| = \frac{\Omega R}{c} B_r
\end{equation}
Therefore the drift velocity of the plasma is can be calculated from $E$ cross
$B$, and find the Lorentz factor
\begin{equation}
  \label{eq:3}
  \frac{v}{c} = \frac{E}{B}, \quad\gamma^2 = \frac{1}{1 - v^2/c^2} = 1 + (\Omega R/c)^2
\end{equation}
This means that near the stellar surface, the Lorentz factor will be around 1,
but outside the light cylinder the Lorentz factor scales linearly with distance.

However this kind of acceleration can't continue forever, because the outflow
has mass attached to it. Lets look at the conserved quantities along jets, which
are ratios of conserved fluxes. One conserved quantity is
\begin{equation}
  \label{eq:4}
  \eta = \frac{F_M}{F_B} = \frac{\gamma \rho v_p}{B_p} = \mathrm{const}
\end{equation}
which is the ratio of the magnetic flux to the mass flux. The second is the
ratio of the energy flux over mass flux
\begin{equation}
  \label{eq:5}
  \mu = \frac{F_E}{F_M} = \gamma\frac{F_{EM}}{F_{KE}} + \gamma
\end{equation}
The ratio $F_{EM}/F_{KE}$ is called the magnetization $\sigma = b^2/4\pi\rho
c^2$. This means that $\mu$ sets the maximum Lorentz factor. Also $\sigma$ sets
the speed of the fast waves $\gamma_F = \sigma^{1/2}$. In force-free the fast
waves travel at speed of light. In reality when the jet Lorentz factor catches
up with this fast wave, force-free approximation breaks down
\begin{equation}
  \label{eq:6}
  \gamma = \gamma_F = \sigma^{1/2} = (\mu/\gamma)^{1/2},\quad \gamma = \mu^{1/3}\ll \mu
\end{equation}
Because acceleration cease to be linear after a certain radius, we can't really
get to the limit $\gamma \sim \mu$, but way less than that.

In 2D $\gamma$ doesn't saturate exactly at $\mu^{1/3}$, but at
\begin{equation}
  \label{eq:7}
  \gamma \propto \mu^{1/3}\log^{1/3}r
\end{equation}

Our key assumption above was that $B_{\phi}^2 - E^2 \ll B_r^2$. Because of this,
toroidal magnetic pressure contribution breaks force balance, bending the field
lines, and fast jets can't make sharp turns. In the opposite limit we ignore
$B_r$, we have
\begin{equation}
  \label{eq:8}
  \gamma^2 = \frac{B_{\phi}^2}{B_{\phi}^2 - E^2}
\end{equation}
If we consider the force in more detail, one force is the pressure gradient $F_m
= -\nabla p_m = p_m/R$. The other is the centrifugal force $F_{c} =
\epsilon_m\gamma^2/R_{c}$. If we want these forces to be equal, then
\begin{equation}
  \label{eq:9}
  \gamma = \left( \frac{R_{c}}{R} \right)^{1/2}
\end{equation}
Now we can combine both limits
\begin{equation}
  \label{eq:10}
  \frac{1}{\gamma^2} = \frac{1}{\gamma_1^2} + \frac{1}{\gamma_2^2}
\end{equation}
where $\gamma_{1}$ and $\gamma_{2}$ are the two limits we got in the two
different extremes.

Now lets look back at the fluxes. The electromagnetic energy flux is
\begin{equation}
  \label{eq:11}
  F_{EM} = \frac{cEB_{\phi}}{4\pi} = \frac{\omega^2R^2B_{\phi}^2}{4\pi c}
\end{equation}
we can rewrite $\mu$ as
\begin{equation}
  \label{eq:12}
  \frac{\Omega^2}{4\pi^2\eta c}\pi B_pR^2 + \gamma = \frac{\mu}{\Phi}\pi B_pR^2 + \gamma
\end{equation}
where $\Phi$ is the magnetic flux $\Phi = (\pi B_pR^2)_F$. Therefore there will
be no acceleration if the magnetic flux remains at large distances:
\begin{equation}
  \label{eq:13}
  \frac{\gamma}{\mu} = 1 - \frac{\pi B_pR^2}{\Phi}
\end{equation}

How do hydro accelerate fluids? We need to make a nozzle, and in MHD we need
something similar. We need magnetic flux bunching toward jet axis.

Outside the fast point $\gamma_{F}$ all signals travel inside the cone with
opening angle
\begin{equation}
  \label{eq:14}
  \xi = \frac{\gamma_F}{\gamma} \approx \frac{\sigma^{1/2}}{\gamma}
\end{equation}
For communication across jet requires that $\theta \lesssim \xi$. Thus the
acceleration limit can be relaxed
\begin{equation}
  \label{eq:15}
  \gamma \lesssim \frac{\mu^{1/3}}{\theta^{2/3}}
\end{equation}
Therefore jets accelerate much better near the axis!

However, most jets are collimated and are not monopolar. How do collimated jets
accelerate? Collimated jets by its geometry has to keep in causal contact. All
signals travel inside the Mach cone
\begin{equation}
  \label{eq:16}
  \xi \leq \frac{\gamma_F}{\gamma} = \frac{\sigma^{1/2}}{\gamma}
\end{equation}
We expect that $\gamma\theta \lesssim \sigma^{1/2} \lesssim 1$. In AGN jets this
is observed, but in GRBs we have $\gamma\theta \sim 10-100$. What does this
mean? One possibility is that GRB jets are unconfined after a certain radius
which could be the surface of a star. When the magnetic field lines are
unconfined they can expand and create bigger pressure gradient.





\end{document}
