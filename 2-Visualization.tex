\documentclass[letterpaper, 11pt]{article}

\usepackage[pdftex]{graphicx}
\usepackage{epstopdf}
\DeclareGraphicsRule{*}{mps}{*}{} 

\usepackage{amsmath, amsthm, amssymb}
\usepackage{listings}
\usepackage{float}
\usepackage{enumerate}
% \usepackage{mystyle}
\usepackage{hyperref}
\usepackage{tikz}
\usepackage{fancyheadings}
\usepackage{tensor}
\usepackage{mathrsfs}
\usetikzlibrary{positioning}
\usetikzlibrary{decorations.pathmorphing}
\usetikzlibrary{arrows}
\usetikzlibrary{decorations.markings}
%\usepackage{fullpage}
\usepackage[left=0.75in, top=1.25in, right=0.75in, bottom=1.25in]{geometry}
\newcommand{\lambdabar}{{\mkern0.75mu\mathchar '26\mkern -9.75mu\lambda}}

\numberwithin{equation}{section}
\numberwithin{figure}{section}

\begin{document}

\title{Visualization Analysis and Design}
\author{Tamara Munzner}
\date{July 18, 2016}

\maketitle

\section{Lecture 1}
\label{sec:lec1}

Disclaimer: taking notes for lectures which use slides is harder because I can't
take advantage of the time the speaker writing on the board. So this will
probably contain a lot less information than the previous one. Fortunately the
notes by the speaker is available
\href{http://www.cs.ubc.ca/~tmm/talks.html#vad16pitp}{here}.

Lets start with the definition of visualization. Computer-based visualization
systems provide visual presentatinos of datasets designed to help people carry
out tasks more effectively. When there is fully automated solutions,
visualization is not very useful, but when we have an ill-specified problem, we
need to use visualization to help us understand the problem better to build the
model. It could also help developers and end users of automatic solutions to
determine parameters / build trust. Visualization is a way to help us out of our
jobs: towards ultimate automatic solutions.

But why do we use an external representation? If we have a table of numbers it
is hard to keep track of them in our minds, but if we have a short-cut through
the cognitive act of memorizing and analyzing numbers by plotting them, then we
can recognize patterns and understand trends much more easily. We don't want to
burden our cognition with low-level book-keeping, but save them for much higher
level understanding, and that is where visualization comes in.

How does visualization help people do things faster? It summarizes lose
information, details matter. We can put things together to confirm expected and
find unexpected patterns, etc. The so-called Anscombe's Quartet shows 4 datasets
with identical statistics, but look completely differently when plotted. These
kind of situations are where it is very useful to show the whole dataset at the
same time without losing details.

We can build an analysis framework by putting it under 4 levels and ask 3
questions. The first level is \emph{domain}, i.e. ``who are the target users''?
There is a vocabulary for a domain (domain specific language), to communicate
with the users of that specific domain. Then we come down to the next level
which is \emph{abstraction}. Here we want to abstract/translate the specifics of
domain to the vocabulary to visualization. We don't usually draw exactly what we
are given, but transform them to a new form. This is the question of ``What is
shown?''. We also need to address ``Why is the user looking at this''.

The next level is \emph{idiom}, i.e. ``how to draw the visualization and how to
manipulate them?''. Then the last level is the \emph{algorithm} level which is separate
from the representation levels above.

We have different ways to get things wrong at any level. At the domain level, we
could assume the needs of the audience wrong, e.g. assuming the same level of
domain-specific knowledge as the speaker from the audience. At the abstraction
level we could be showing the wrong data. At the visual level we could be
showing things in a way that doesn't work, and finally we may have a code too
slow to do the job.

How to do things right? We need to use methods from different fields at each
level. At algorithm level we are talking about computer science, which is
technique-driven work. At idiom level we are in the regime of design and
cognitive psychology. At the higher levels we start to venture into the
field of anthropology/ethnography. Here we are talking about problem-driven
work.

Lets start by asking the question of \emph{What}. There are three main types of
datasets: tables, networks, and spatial (fields or geometry). Tables are easy to
understand, with columns and rows. Networks are like graphs with nodes and links
between them. Spatial data can come as fields, like a grid of positions with
values associated to them. It could also be geometry, like a map.

There are also different types of attributes that we want to represent. We
could have categorical data, or those with intrinsic ordering. With the latter
we also have the question of which ordering direction
(sequential/diverging/cyclic). In situations where the data is cyclic it could
represent some interesting challenges.

Lets now look at the question of \emph{Why}. We can split the question into
actions and targets. On the action side, we typically want to analyze/query
data. Analyzing is a way to consume data, to discover things or present them to
others. We could also present new data by annotating older sets or by deriving
new things from older things. Querying is changing the amount of data, to
identify/compare things, typically reducing the amount of data to process in our
head.

We can talk a bit more about deriving. We actually rarely draw things we are
given directly! First we need to decide what the right thing to show is, create
it with a series of transformations on the original data, and then draw that. A
simple example is a graph of exports and imports, we could take the difference
and plot the trade balance.

An example of analysis is to derive one attribute from a series of
trees/networks, namely the Strahler number. Details in the slides, I don't want
to summarize it on the fly\dots

Now lets talk about the targets. We might want to present trends, outliers, or
features, where the last one is an all-encompassing way to say things for
example quantitative\dots

Lets now talk about the important question of \emph{How}. First thing we need to
do is to encode data. There are various structures for visually encoding the
data. To talk about this, we introduce marks and channels. Marks are geometric
primitives, and channels control the appearance of marks, e.g. color, position,
size, etc. We can use different channels to encode the same data to bring
emphasis to the data, or we could add different channels to encode different
content to squeeze in more information.

The different idiom structures of visually encoding the data are combinations of
marks and channels. A histogram is using the vertical positions of lines to
encode quantities. A scatter plot is using the vertical/horizontal positions of
points, etc.

There are two main types of channels, one is magnitude, and the other is
identity. The magnitude channel, like position, length, saturation, are suited
at showing ordered attributes, whereas the identity channel (spatial region,
shape, etc.) is suited at showing categorical attributes. We need to match the
channel and the data attributes we want to represent.

Different channels have different effectiveness! We are very good at judging
spatial positions, but not very good at judging saturation or luminance. We want
to encode the most important attributes with the highest ranked channels.
Spatial position ranks the highest for both magnitude and identity channels, so
it should be the first thing we think about when we determine the way to encode
data. One thing to keep in mind is that there are people who respond less well
to color hues (colorblindness).

Where does the channel ranking come from? We have the Steven's Psychophysical
Power Law: $S = I^N$. The perceived sensation of length has index $N=1$, with
area/depth at $N=0.7/0.67$. There were visualization experiments done by
Cleveland \& McGills, and experiments using crowdsourcing.

In addition to the channel to choose, we need to consider how many usable steps
we have at each channel. We need to have a channel with enough different levels
to show the different data we need. We also can't use all the channels at once,
because sometimes channels step on each other. Refer to slide page 32 for a
comparison when we have channels that interfere with each other.

One thing we need to consider is ``Popout''. When we have multiple clues for
popout the data basically presents itself, but when we need to do a conscious
search on the data representation it takes a much longer time on our cognitive
system to recognize the pattern.

We can represent grouping between features. We can use containment or
connection, or just put related things close to each other.

One last word about representation is relative vs.\ absolute judgments.
The perceptual system mostly operates with relative judgments, not absolute one.
Therefore to represent the comparison of data, accuracy increases with common
frame/scale and alignment. The ratio of increment vs.\ common background
determines the ease of judgment.

Now lets to talk how to use space to encode each of the data types. The first is
\emph{tables}. Lets use the computer science words key and values to denote the
table. The key is an independent attribute to uniquely index the items we want
to look up. A simple table has one key, whereas multi-dimensional table has
multiple keys. Values are simply the data we want to represent.

As an example, the scatter plot is used to express values only, with no keys. We
simply represent two quantitative attributes on each dimension. It scales very
well since we can fit in thousands of points on a plot. A bar chart has one key
and one value. The key is a category attribute and the value is a quantitative
attribute. It is good at comparing or looking up values. A line chart also has
one key and one value, with point as the mark. Now the key can be a quantitative
attribute.

When should we should bar charts vs.\ line charts? The answer depends on the
type of the key attribute. For female/male key (categorical) it is better to use
bar charts, and for a number key (quantitative) it is better to use line chart.

Lets consider heatmap. We have two keys, x-y position, and one value as color.
The two keys can be categorical attributes, and the value is quantitative. It
can be useful at finding clusters and outliers. However we are limited by how
much display can resolve on the screen, and we are capped at $\sim
1\,\mathrm{M}$ items. We are however very limited in the quantitative attribute
levels since we only have so many colors/regions to play with without
interfering with each other.

We can combine different idioms together using for example scatterplot matrix,
where we have a matrix of scatter plots put together. We can also have parallel
coordinates that put different line charts together. Either of these we are
limited to dozens of attributes.

We also have pie charts or polar area charts. In the former area marks with
angle channel to represent data, and in the latter we have length channel for
the area marks. Both are better replaced by bar charts. One argument for pie
charts is to show parts-to-whole idea, but we could fix that by plotting a
normalized stacked bar chart.

There are interesting cases where one could use radial orientation to represent
data. One example is glyphmaps to represent cyclic data.

Let's talk about spatial data now. We often want to use directly the data we are
given, since the data is encoded directly in the position they are given. For
example the choropleth map is to color-code the different regions of a map to
represent a quantitative attribute on a geographic geometry. One important issue
here however is the normalization of data.

We could also have a topographic map, which is plotting a scalar spacial field
on geographic geometry using contours. Similar idea is to plot isosurfaces or
directly volume rendering which is to plot scalar spatial field directly in a 3D
space by rendering the isosurface of the value. Due to the 3D nature often we
often use opacity as well as colors to encode the desired data.

There are a family of ways to show vector and tensor fields. The challenge is
that there are many attributes to show per cell. We could have flow glyphs where
we just show local directions. We could also have geometric flow that trace the
trajectory of particles. We could have texture flow and feature flow. In general
there is no single best way but it depends on the task we want to carry out.

The last data type is networks, which is when we add links to the dataset. One
way is simply use node/lines to directly show the networked graph. This is good
for exploring topology, locating paths and clusters, etc. It is tricky however
to consider spatial positions and proximity semantics, since we may be conveying
unintended information. The scalability of this kind of graph is limited by the
number of edges. This kind of graph only makes sense when we have number of
nodes $N$ less than 4 times the number of edges $4E$.

We could also transform the network into a heatmap where we encode two
categorical attributes with x-y positions and use color to represent link.
However we need to train ourselves to recognize topological structure. The
strength of this representation lies in the scalability.

We could also use a radial node-link tree. It is easy to understand topology and
to follow paths.


This feels like a shitty essay to type out. I'm not going to take notes for the
second part\dots



\end{document}
