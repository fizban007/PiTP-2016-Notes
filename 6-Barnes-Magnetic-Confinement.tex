\documentclass[letterpaper, 11pt]{article}

\usepackage[pdftex]{graphicx}
\usepackage{epstopdf}
\DeclareGraphicsRule{*}{mps}{*}{} 

\usepackage{amsmath, amsthm, amssymb}
\usepackage{listings}
\usepackage{float}
\usepackage{enumerate}
% \usepackage{mystyle}
\usepackage{hyperref}
\usepackage{tikz}
\usepackage{fancyheadings}
\usepackage{tensor}
\usepackage{mathrsfs}
\usetikzlibrary{positioning}
\usetikzlibrary{decorations.pathmorphing}
\usetikzlibrary{arrows}
\usetikzlibrary{decorations.markings}
%\usepackage{fullpage}
\usepackage[left=0.75in, top=1.25in, right=0.75in, bottom=1.25in]{geometry}
\newcommand{\lambdabar}{{\mkern0.75mu\mathchar '26\mkern -9.75mu\lambda}}

\numberwithin{equation}{section}
\numberwithin{figure}{section}

\begin{document}

\title{\#MCF: The Physics of Magnetic Confinement}
\author{Michael Barnes}
\date{July 20, 2016}

\maketitle

\section{Lecture 1}

We are going to talk about magnetic confinement in fusion. This is our only talk
which is not on Astrophysics in this school, so lets start by talking about how
this differ from astrophysical plasmas. The first difference is that, for the
sun as an example, it has a great mass and the gravitational force can provide
confinement for the plasma. However for controlled fusion we don't have such
luxury, and we need magnetic or hydro means to confine the plasma.

The way we do fusion is to heat up hydrogen to the extent such that at the tail
of the distribution the atoms have just enough energy to tunnel through the Coulomb
barrier to undergo fusion. Therefore our goal is to confine hot plasma. We use a
toroidal structure and try to achieve an equilibrium state, which we will talk
about subsequently.

We are not going to talk too much about how to heat up the plasma. In a short
sentence we just put current into the plasma and due to resistivity it will heat
up.

\subsection{Single Particle Confinement}

The basic concept of magnetic confinement is simple. With magnetic field
particles will stream along the magnetic field lines while gyrating at radius
\begin{equation}
  \label{eq:1}
  \rho = v_{\perp}\Omega \propto v_\mathrm{th}/B \sim \sqrt{\beta}
\end{equation}
However this only works with 2 dimensions and particles can still stream along
the field line.

One idea to overcome this is to use a magnetic mirror, briefly covered in Matt
Kunz's lecture. This is what happens in Van Allen radiation belt around Earth.
Since magnetic moment is an adiabatic invariant in slowly changing inhomogeneous
magnetic field, particles will bounce off regions with higher magnetic field.
Since energy is conserved
\begin{equation}
  \label{eq:2}
  \frac{m v_{\parallel}^2}{2} = E - \mu B
\end{equation}
therefore at bounce point $E = \mu B_\mathrm{max}$. However this method of
confinement can lose a lot of plasma when it is hot.

Another solution is to use the field line topology to confine the plasma. The
``Hairy ball theorem'' tells that the only topology with confined trajectories
is a torus. Once we confine ourselves to torus, there are three possibilities:
closed lines, surfaces, or toroidal annuli. The third idea is not very appealing
especially when there are many particles, since particles on different field
lines will interact.

If we have a circular magnetic field in a plane, particles gyrating in the
magnetic field will drift vertically due to grad-B drift. If we put $E$ field to
prevent this movement, the particles will drift outwards due to E cross B drift.
So it does not work.

The solution is to twist the magnetic field by introducing poloidal
$B_{\theta}$. This is the Tokamak. We can use a toroidal coil to provide the
toroidal magnetic field and use a coil inside and outside to make poloidal
magnetic field. In this kind of configuration, if the particles sample the
entire toroidal surface, then there is no net drift.

However there is a complication. The field strength changes over the particle
trajectory and some particles might get trapped. At this point lets introduce
some quantities
\begin{equation}
  \label{eq:4}
  \psi_p = \oint \mathbf{B}\cdot d\mathbf{a}^{\theta},\quad \psi_t = \oint \mathbf{B}\cdot d\mathbf{a}^{\zeta}
\end{equation}
where $\zeta$ labels the coordinate along the torus long direction, and $\theta$
labels the short direction (orthorgonal).



\subsection{Magnetic Topology}

Tamm's Theorem states that no average radial drift in axisymmetric torus. If the
confining field is axisymmetric, the canonical angular momentum is conserved
\begin{equation}
  \label{eq:5}
  p_{\zeta} = Rmv_{\zeta} + \frac{ZeRA_{\zeta}}{c} = \mathrm{const}
\end{equation}
Trapped particles in an axisymmetric torus will close in their orbits and
precess along the field lines. 

There are other symmetries that also prevent average radial drift. Lets consider
\begin{equation}
  \label{eq:6}
\psi_{*} = \psi_p - \frac{I(\psi_p)v_{\parallel}}{\Omega}
\end{equation}
where $I\sim B/R$ is small. We want to argue that
\begin{equation}
  \label{eq:7}
  \frac{d\psi_{*}}{dt} = \frac{2v_{\parallel}^2 + v_{\perp}^2}{\Omega B^2}\left( \mathbf{B}\times\nabla\psi_p\cdot\nabla B - I\mathbf{B}\cdot\nabla B \right)
\end{equation}

To summarize, the magnetic field topology will be
\begin{equation}
  \label{eq:8}
  2\pi \mathbf{B} = \nabla \psi_t \times \nabla \vartheta + \nabla\zeta\times\nabla\psi_p(\psi_t, \vartheta, \zeta)
\end{equation}
The trajectory of the field line is given by
\begin{equation}
  \label{eq:9}
\frac{dx^i}{d\tau} = \mathbf{B}\cdot\nabla x^i = B^i
\end{equation}
We want to choose $\zeta$ as time-like cooordinate:
\begin{equation}
  \label{eq:10}
  \frac{d\psi_t}{d\zeta} = -\frac{\partial\psi_p}{\partial\vartheta},\quad \frac{d\vartheta}{d\zeta} = \frac{\partial\psi_p}{\partial\psi_t} = 1/q
\end{equation}
which is the pitch of the magnetic field.

These two equations can be identified as Hamilton equations of this system where
\begin{equation}
  \label{eq:11}
  \psi_p \leftrightarrow H,\quad \zeta\leftrightarrow t
\end{equation}
where $\theta$ and $\psi_t$ are position and momentum. This Hamiltonian system
has one degree of freedom which allows 1D trajectories that are closed lines,
and 2D trajectories that wraps around the torus.

Now we have two possibilities, tokamaks and stellarators. Both are topologically
tori, but Tokamaks are easier to construct with confined drift orbits. However
in stellarators the confining poloidal field is generated externally and it is
harder to confine the orbits. But this system has more flexibility in shaping.

\subsection{Plasma Equilibrium}

The force on the plasma is
\begin{equation}
  \label{eq:12}
  \frac{\partial (m_sn_s\mathbf{u}_s}{\partial t} + \nabla\cdot \mathbf{P}_{s} - e_sn_s \left( \mathbf{E} + \frac{\mathbf{u}_s\times \mathbf{B}}{c} \right) = \mathbf{F}_s
\end{equation}
Since we want to find equilibrium without external forces, both the first term
on lhs and the rhs are zero. Poisson's equation tells us that
\begin{equation}
  \label{eq:13}
  \nabla^2\Phi = -4\pi\sum_se_sn_s,\quad \sum_se_sn_s\sim en_{e}\left( \frac{\lambda_D}{L} \right)^2
\end{equation}
As long as the plasma is quasineutral, with density variation much smaller than
the length scale of the system, then the static electric field can be neglected.
Near thermal equilibrium we have $\nabla\cdot \mathbf{P}_s = \nabla p_s$. So we
have after summing over all species
\begin{equation}
  \label{eq:14}
  c\nabla p = \mathbf{J}\times \mathbf{B}
\end{equation}
What this does tell us is that in a collisionless limit particles propagate
parallel to the field lines, and in the collisional limit we have sound waves
across the field lines.

Another consequence of quasineutrality is $\nabla\cdot \mathbf{J} = -\partial
\rho / \partial t$, so
\begin{equation}
  \label{eq:15}
  \nabla\cdot \mathbf{J} \sim \frac{\rho}{\tau} \sim \frac{J}{L}\left( \frac{v_{th,e}^2}{c} \right)^2\frac{1}{\Omega_{e}\tau} \ll \frac{J}{L}
\end{equation}
Therefore mainly $\mathbf{J}_\perp$ enters the equation before. This means there
has to be some parallel current $\nabla\cdot J_{\parallel} \neq 0$.

There is an alternative physical interpretation. Lets define the curvature
\begin{equation}
  \label{eq:16}
  \kappa = \mathbf{b}\cdot\nabla \mathbf{b} = \frac{4\pi}{c}\frac{\mathbf{J}\times \mathbf{B}}{B^2} + \frac{\nabla_{\perp}B}{B}
\end{equation}
If we use this to replace the J cross B term we have
\begin{equation}
  \label{eq:17}
  \frac{B^2}{4\pi}\kappa = \nabla_{\perp} \left( p + \frac{B^2}{4\pi} \right)
\end{equation}
This means if we want to stretch field lines it has some tension to oppose that,
and it balances plasma pressure.

Lets look at a simplified version of axisymmetric equilibrium. We look at the
radial component of Ampere's Law
\begin{equation}
  \label{eq:18}
  \frac{4\pi}{c}J^r = \mathcal{J}\left( \frac{\partial B_{\zeta}}{\partial \theta} - \frac{\partial B_{\theta}}{\partial\zeta} \right) ,\quad \mathcal{J}^{-1} = \nabla r\cdot(\nabla\theta\times\nabla\zeta)
\end{equation}
where $r$ is some variable that labels the field lines. This tells us that
$B_{\zeta}$ only depends on flux label $r$, so we can write $B$ field as
\begin{equation}
  \label{eq:19}
  \mathbf{B} = I(r)\nabla\zeta + \nabla\zeta \times \nabla \psi_p
\end{equation}
We can plug this into our force balance to get
\begin{equation}
  \label{eq:20}
  cp' = \mathcal{J}^{-1}B^{\theta}(qJ^{\theta} - J^{\zeta})
\end{equation}
and we can find some expressions for $J^{\theta}$ and $J^{\zeta}$. We can find
the Grad-Shafranov equation
\begin{equation}
  \label{eq:21}
  R^2\nabla\cdot(R^{-2}\nabla\psi_p) = -I\frac{dI}{d\psi_p} - 4\pi R^2\frac{dp}{d\psi_p}
\end{equation}

If we use cylindrical coordinates $(R, Z)$ and define $\psi_p = u\sqrt{R}$, then
we can find a Poisson equation for $u$, given $p(u)$, $I(u)$ and some boundary
condition we can solve it iteratively.

Lets look at one way to solve it numerically. The idea is to map the constant
flux surfaces conformally to a circle, and transform the Poisson equation
accordingly. Now because this is a periodic domain we can solve the equation
using spectral method. On a computer we can use FFT to quickly solve the
equation using Fourier transform. However when we do the conformal map back to
the original domain, there might be bunching of grid points or region with
relatively few grid points, so one needs to be careful.

We can also use similar analysis to give Grad-Shafranov equation for the
parallel current:
\begin{equation}
  \label{eq:22}
  J_{\parallel} = \frac{-c}{4\pi}B\frac{dI}{d\psi_p} - \frac{cI}{B}\frac{dp}{d\psi_p} = \sum_se_s\int d^3v\,v_{\parallel}F_{s}
\end{equation}
The occurance of distribution function necessitates kinetic treatment.

\subsection{Stability}

In this kind of plasma there is a wide zoo of plasma instabilities. Due to the
toroidal geometry it is difficult to decouple the modes, so it is a huge mess.
Lets only talk a bit about the most unstable modes.

When we solve the equilibrium problem we saw there are a number of different
forces on the plasma. We have field line ``tension'' which opposes the bending
of magnetic field lines. The most dangerous mode is due to opposing the bending
field line which has $k_{\parallel} \ll k_{\perp}$. There is the distinction of
``rational'' field lines which closes themselves, and ``irrational'' field lines
which don't. Only on rational field lines we can have $k_{\parallel} = 0$.
Flute perturbations are unstable when the field line curves towards the plasma,
but it its stable when the field line curvature is away from plasma.

Tokamak has good and bad curvature regions. Field lines will bend outwards in
the bad curvature region to somewhat self-stabilize. If we have field-aligned
perturbations at a given radius, but different pitch of magnetic field lines at
neighboring radius it could also stabilize the perturbation.

Another kind of instability is kink instability. Kinks come from helical
perturbation and is current driven. If we kink a magnetic field line bundle like
a hose, the upper part of the kink has less magnetic pressure so it is even
easier to kink upwards, which runs away. Again field line bending can stabilize
this effect.

In theory if we want to find out whether the plasma is stable, we want to
linearize the MHD equations and solve the linear problem as an eigenvalue
problem. However in Tokamak plasmas all the modes couple, so it is not as easy
to do the same thing. We use the so-called energy principle by looking at the
energy and work of the displacements
\begin{equation}
  \label{eq:23}
  \delta W (\xi^{*}, \xi) = -\frac{1}{2}\int d^3r \xi^{*}\cdot \mathbf{F}(\xi),\quad E(\xi^{*},\xi) = \frac{1}{2}\int d^3r\rho_0\dot{\xi}^{*}\cdot\dot{\xi}
\end{equation}
and we use variational principle on
\begin{equation}
  \label{eq:24}
  \Omega^2 = \frac{\delta W}{E},\quad \delta\Omega^2 = 0
\end{equation}
If we expand the displacement into discretized modes, then we find
\begin{equation}
  \label{eq:25}
  \delta W = \sum_n \left| a_n \right|^2\omega_n^2
\end{equation}
When $\delta W < 0$ we have an instability, and if $\delta W\geq 0$ for all
displacements then we have stability. The complication here is that in MHD we
have a continuous spectrum of modes, so we need to do more work but the
principle is similar.

In practice the problem is reduced to numerically minimizing the function
\begin{equation}
  \label{eq:26}
  \delta W = \sum_n\sum_ma_na_m\mathbf{F}(\xi_n,\xi_m)
\end{equation}

I honestly don't understand what I'm writing\dots


\end{document}
