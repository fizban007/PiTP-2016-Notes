\documentclass[letterpaper, 11pt]{article}

\usepackage[pdftex]{graphicx}
\usepackage{epstopdf}
\DeclareGraphicsRule{*}{mps}{*}{} 

\usepackage{amsmath, amsthm, amssymb}
\usepackage{listings}
\usepackage{float}
\usepackage{enumerate}
% \usepackage{mystyle}
\usepackage{hyperref}
\usepackage{tikz}
\usepackage{fancyheadings}
\usepackage{tensor}
\usepackage{mathrsfs}
\usetikzlibrary{positioning}
\usetikzlibrary{decorations.pathmorphing}
\usetikzlibrary{arrows}
\usetikzlibrary{decorations.markings}
%\usepackage{fullpage}
\usepackage[left=0.75in, top=1.25in, right=0.75in, bottom=1.25in]{geometry}
\newcommand{\lambdabar}{{\mkern0.75mu\mathchar '26\mkern -9.75mu\lambda}}
\newcommand{\Jmath}{J}

\numberwithin{equation}{section}
\numberwithin{figure}{section}

\begin{document}

\title{Selected Topics in Plasma Astrophysics}
\author{Eliot Quataert}
\date{July 19, 2016}

\maketitle

\section{Lecture 1}

\subsection{Astrophysical Plasmas in General}

Plasma astrophysics is a very broad subject, encompassing a lot of different sub
fields. There are a lot of different techniques to study them as well. For
relativistic plasma physics, there are force-free electrodynamics to study e.g.\
pulsar magnetospheres. There is a family of (GR)(M)HD that study accretion,
jets, etc. There is PIC that is used to study shocks, and Dynamic space-time +
MHD to study compact object mergers.

For non-relativistic theory we use force-free for e.g.\ solar corona, MHD for
e.g.\ star formation, disks, etc., and kinetic theory for shocks, reconnection,
turbulence, etc. Even within MHD, there is a wide range of fluid models which
are useful for different situations.

In the first lecture we focus on fluid models with some kinetic elements.

\subsection{Stellar Winds}

Stellar winds is a topic that is rich and has influence over many other branches
of plasma astrophysics, and it will be our topic today. We will be discussing
different kinds of winds:
\begin{itemize}
    \item Thermally driven winds (from sun-like stars): hydrodynamic theory, kinetic
  theory
    \item Magnetocentrifugically driven winds: MHD, rotation as energy source
      but trapped by $B$ fields.
    \item Radiation pressure driven outflow: $L > L_\mathrm{edd}$: continuum
      driven or line-driven, depending on the spectrum
\end{itemize}

\subsection{Solar Corona and Wind}

Solar corona and outflow is an important part of solar dynamics. The solar
outflow is very small in terms of mass and energy. However the solar wind is
efficiently extracting the angular momentum from the sun. Its spin has slowed
down by a factor of 30 to 50 since its birth. The time scale of $dJ/dt$ is on
the order of $10^{10}$ years.

The solar corona is not in thermal equilibrium. We know that $T_i \gg T_p \geq
T_{e}$, and is anisotropic $T_{\perp}\geq T_{\parallel}$. The corona is mostly
collisionless where $\ell_{mfp}\sim 10^{8}\rho_\mathrm{Larmor}$. It turns out
that despite the collisionlessness we can use fluid theory to somewhat describe
the solar wind.

Lets start by talking about the Parker model. Consider a time-independent
outflow, the continuity equation is
\begin{equation}
  \label{eq:1}
  \nabla\cdot (\rho \mathbf{v}) = 0 \Longrightarrow r^2\rho v_r = \text{const}
\end{equation}
From this the mass loss rate is simply $\dot{M} = 4\pi r^2\rho v_r$ which is
also a constant. The radial velocity equation reads
\begin{equation}
  \label{eq:2}
  v\frac{dv}{dr} = -\frac{1}{\rho}\frac{dP}{dr} - \frac{GM}{r^2}
\end{equation}
If we assume temperature being constant, pressure $P = \rho c_s^2 = \rho
kT/m_p$, then we have this equation
\begin{equation}
  \label{eq:3}
  \frac{1}{v_r}\frac{dv_r}{dr}(v_r^2 - c_s^2) = \frac{2c_s^2}{r} - \frac{GM}{r^2}
\end{equation}
This equation describes spherical wind, or spherically symmetric (Bondi)
accretion.

There is a special point in this equation called the sonic point. This is where
$v_r = c_s$. This requires that the right hand side is also zero
\begin{equation}
  \label{eq:4}
  \frac{2c_s^2}{r} = \frac{GM}{r^2}
\end{equation}

In the regime where $v < c_s$, we can ignore the left hand side of equation
\eqref{eq:3}. This gives a spatial distribution of density. However when this is
true, the pressure is very big. In fact if we require that the flow goes through
the sonic point peacefully, we can uniquely obtain a unique outflow solution
from this equation. TODO: clean this paragraph up

\subsection{Magnetic Field}

Around the sun there are magnetic field lines that open up to infinity, and
field lines that close back to the star itself. For a non-rotating star, the
above theory can be thought of as describing the acceleration of the outflow
along the open magnetic field lines. The only non-trivial thing is the structure
of the global magnetic field.

However now let's consider when the star is rotating. Lets simplify the actual
magnetic field to a split monopole configuration, which can be called a
``theorist's monopole''. This is a useful toy model where we can calculate
structures of the outflow relatively easily. If we have a very powerful outflow
that blows up the magnetic field lines, then at large distances it will actually
look like a split monopole field.

We will now focus on the equatorial plane. Imagine the magnetic field lines have
infinite tension. The outflow will rigidly follow the field lines, and forced to
corotate with the star. As they flow out their rotation velocity will increase
with radius, as well as their specific angular momentum. Therefore they extract
angular momentum from the star. The magnetic field acts as a medium that
extracts angular momentum from the star and transfer it to the outflow.

The corotation only can keep up till some point, where the magnetic energy
becomes comparable to the kinetic energy of the material
\begin{equation}
  \label{eq:5}
  \frac{B^2}{8\pi} \sim \rho v_r^2
\end{equation}
At this point there is so much energy in the gas that you can't treat the
magnetic field as rigid. This is called the \emph{Alfven point} $r_A$. This is also
when
\begin{equation}
  \label{eq:6}
  v_r = v_A \sim \frac{B}{\sqrt{4\pi\rho}}
\end{equation}

Lets look at this outflow in more detail. Let
\begin{equation}
  \label{eq:7}
  \mathbf{v} = v_r\hat{r} + v_{\phi}\hat{\phi},\quad \mathbf{B} = B_r\hat{r} + B_{\phi}\hat{\phi}
\end{equation}
We want to use the conservation of momentum and energy again like before to
write down the equation governing the outflow. We have
\begin{align}
  \dot{M} = 4\pi r^2 \rho v_r = \text{const}, \quad \frac{\partial \mathbf{B}}{\partial t} = \nabla\times(\nabla \times \mathbf{B}) = 0
\end{align}
We can show that
\begin{equation}
  \label{eq:8}
  \frac{B_{\phi}}{B_{r}} = \frac{v_{\phi} - r\Omega}{v_r}
\end{equation}
The way to think about this result is that, if there is perfect corotation, then
there is no $B_{\phi}$. If $v_{\phi}$ can't keep up with $r\Omega$ the magnetic
field will start to lag behind and be swept back, leading to a negative
component of $B_{\phi}$.

The next equation which is important is the momentum equation in the $\phi$
direction. It tells us how does angular momentum change
\begin{equation}
  \label{eq:9}
  \rho v_r\frac{d}{dr}(rv_{\phi}) = \frac{B_r}{4\pi}\frac{d}{dr}(rB_\phi)
\end{equation}
If we multiply by $r^2$ we can easily integrate this equation, giving
\begin{equation}
  \label{eq:10}
  L = rv_{\phi} - \frac{rB_{\phi}B_r}{4\pi\rho v_r} = \text{const}
\end{equation}
There are two contributions to the angular momentum: $L_\mathrm{gas}$ and
$L_\mathrm{mag}$. Therefore the angular momentum is carried out both by the gas
and the magnetic field, and the total angular momentum is a constant in the
problem.

To find what the constant is, we can substitute equation \eqref{eq:8} into this
equation and solve:
\begin{equation}
  \label{eq:11}
  v_{\phi} = r\Omega \left( \frac{v_r^2L/r^2\Omega - v_{Ar}^2}{v_r^2 - V_{Ar}} \right)
\end{equation}
where $v_{Ar}^2 = B_r^2/4\pi\rho$. This has the same structure as before: the
denominator is zero at some point. Here is apparent that the magnetic field
component that determines the Alfven point is the radial point. Therefore at the
Alfven point we have
\begin{equation}
  \label{eq:12}
  L = r_A^2\Omega
\end{equation}
This means that the wind solution determines exactly how much angular momentum
is extracted from the star. This idea is also why the outflow model is so
important in plasma astrophysics. Because the magnetic field can be very strong,
we could have $r_A \gg R_{*}$. The rate of angular momentum loss is
\begin{equation}
  \label{eq:13}
  \dot{\Jmath} = \dot{M} L = \dot{M}r_A^2\Omega
\end{equation}
so the time scale is
\begin{equation}
  \label{eq:14}
  t = \frac{J}{\dot{\Jmath}} \sim \frac{MR_*^2}{\dot{M}r_{A}^2}
\end{equation}
Due to the large ``level arm'' of the Alfven radius, this time scale can be much
smaller than the time scale of mass loss. This efficient extraction of angular
momentum due to magnetic field will occur in other places in plasma astrophysics
as well.

The last thing we want to talk about is the energy carried by the outflow.
Consider hydrodynamics, we have the Bernoulli constant which is essentially
energy per unit mass:
\begin{equation}
  \label{eq:15}
  B_l = \frac{1}{2}v^2 + \phi + h,\quad h = \frac{5}{2}\frac{kT}{m_p}
\end{equation}
where $h$ is called the enthalpy $h = \gamma/(\gamma - 1) kT/m$. In thermally
driven wind, if we heat up the gas it will be able to escape. However in MHD we
have an additional loss of energy in the form of Poynting flux, which also taps
into the rotational energy of the star. We find that
\begin{equation}
  \label{eq:16}
  B_l + \frac{S_r}{\rho v_r} = \text{const}
\end{equation}
where $S_r$ is the radial Poynting flux. Given the magnetic field structure it
is easy to calculate the Poynting flux
\begin{equation}
  \label{eq:17}
  \frac{S_r}{\rho v_{r}} = -\frac{r\Omega B_rB_{\phi}}{4\pi \rho v_{r}}
\end{equation}
If we call this constant $\epsilon$, then it is like the angular momentum flux,
it has a gas component and also a magnetic field component, where in fact
$\epsilon_\mathrm{mag} = \Omega L_\mathrm{mag}$.

Now the question is which of these is more important. Lets roughly estimate it
to get a feeling:
\begin{equation}
  \label{eq:18}
  \epsilon_\mathrm{gas} \sim \frac{kT_\mathrm{corona}}{m_p},\quad \epsilon_\mathrm{mag} \sim r_A^2\Omega^2
\end{equation}
For the sun, the gas energy flux is much larger than the magnetic energy flux,
and most of the energy outflow is thermal energy. This is a thermally driven
wind. In the other regime most of the energy that goes to infinity is the
rotational energy of the central object. This is what we call a
magnetically driven outflow. Which regime we have depends no the
temperature of the corona, rate of rotation, and Alfven radius.

Lets come back to the question is why do we use a fluid model at all for the
sun. The sun is a slowly rotating object. Around the sun the presence of
magnetic field means $\rho_\mathrm{Larmor} \ll R$, so plasma streams only in one
direction. Along the $B$ field pressure is the origin of acceleration, but fluid
theory is okay within order of about a few. Kinetic instabilities also helps us
by limiting how much our distribution function can deviate from Maxwellian.

The state of the art of solar corona study is to understand how particles are
heated in the solar corona. This heating has been measured from the corona all
the way to much further away from the Earth. The temperature profile is
shallower than adiabatic, so there must be heating going on. One idea is that
Alfven waves are launched and the turbulence keep heating the plasma.

The time-reversal of this outflow wind model is the spherical Bondi accretion.
There is also the theory of magnetically directed accretion to the central object
analogous to the magnetically driven wind models. But it works not as good
because in the wind model the outflow is driven by the rotation of the star, and
same can't be said for accretion models.

\subsection{Radiative Driven Winds}

Lets now briefly talk about radiation pressure driven winds. Around RGB and AGB
stars, a lot of dust forms in stellar atmosphere which have high $\kappa$. They
feel the radiation pressure much higher than the gas but are coupled to the gas
through collision, therefore driving a wind. Around massive stars $L >
L_\mathrm{edd}$ on metal lines, and the plasma absorption of metal line photons
drives the wind.

In thermally driven winds the energy outflow is $\dot{M}$ times the speed of
sound squared. However for the radiation pressure driven winds we should think
of momentum flux due to photons
\begin{equation}
  \label{eq:19}
  \dot{P} \sim \dot{M}v_{\infty} \sim L/c,\quad v_{\infty} \sim v_\mathrm{escape}
\end{equation}
We can derive this from the momentum equation directly
\begin{equation}
  \label{eq:20}
  v\frac{dv}{dr} = -\frac{1}{\rho}\frac{dP}{dr} - \frac{GM}{r^2} + \kappa \frac{F}{c}
\end{equation}
where the right hand side has a new term which is an average radiation pressure
force over all wavelengths. If we only focus on the new term, we can write
\begin{equation}
  \label{eq:21}
  v\frac{dv}{dr} = \frac{\kappa L}{4\pi r^2 c}
\end{equation}
Multiplying both sides by $4\pi r^{2}\rho$ we have
\begin{equation}
  \label{eq:22}
  \dot{M}\frac{dv}{dr} = \rho\kappa\frac{L}{c}
\end{equation}
Integrating this over all radii to infinity we now have
\begin{equation}
  \label{eq:23}
  \dot{M}v_{\infty} = \tau\frac{L}{c}
\end{equation}
where $\tau$ is the optical depth $\tau = \int \rho\kappa\,dr$. That is exactly
we have, and the optical depth is the one uncertain numerical coefficient.

For line-driven winds refer to (\href{http://adsabs.harvard.edu/abs/1970ApJ...159..879L}{Lucy \& Solomon 1970}; Castor, Abott, Klein 1975).

For applications of this wind theory in other models/fields, refer to the
original slides! Mentioned models are thermally driven galactic winds, line
driven winds from accreting black holes, magnetized winds from accretion disks.

\section{Lecture 2}

\subsection{Instabilities in Ideal Fluids and Dilute Plasmas}

Last time we used the outflow from massive stars to introduce different
techniques and methods that show up in different physical contexts. Today we
will do the same on another topic.

\subsection{Instabilities in Ideal Fluids}

Who cares about linear instabilities when we have simulations? We will argue
that even in the present days the study of linear instabilities is still very
useful. We can't simulate everything, so we need to know what physics to include
in the first place. It is very instructive to identify the key physics problems
of interest and study of linear instabilities is a way to approach that.

Linear instabilities are also a way to produce turbulent transport of mass,
momentum, energy, etc. The physics of linear instabilities is often still there
even when the system has already evolved into the nonlinear state. Linear
instabilities can also fundamentally change the equilibrium structure and
dynamics of the system. For example the FRW universe is unstable to
gravitational perturbations and it leads to the large scale structures we know
today. Another example is the convection in stars. If we write down the
equilibrium theory of stars we will get everything wrong, because the
equilibrium theory relies on diffusion but a large portion of the star like the
sun is convective.

Ideal single fluid (M)HD is a useful starting point for astrophysical plasmas.
It encapsulates the conservation of key quantities like mass, energy, momentum,
and does better than expected. However non-ideal effects and multi-fluid effects
can be very important in many systems. For example in star formation and planet
formation we have dust in addition to multi-fluid MHD. In intracluster plasma in
the galaxy the plasma is very dilute and near collisionless so we need to
consider anisotropic conduction, etc. In luminous accreting black holes we need
to consider radiation pressure effects in addition to MHD. It is because of
these diversities that we need to study the system in the linear regime first
before we dive into simulations directly.


\subsubsection{Buoyancy}

The basic setup of considering whether a system is stable under buoyancy is as
follows. Suppose we have a gravity field and a downwards entropy gradient, it is
a typical system for convection: $ds/dz < 0$. This is called the Schwarzschild
criterion. This is however only a specific criterion. The motion is usually slow
and adiabatic, with time scale
\begin{equation}
  \label{eq:24}
  t_\mathrm{sound} \sim \mathrm{hr} \ll t_\mathrm{buoyancy} \sim \mathrm{month} \ll t_\mathrm{diffusion} \sim 10^{4}\,\mathrm{yr}
\end{equation}

If we have a blob of fluid that flow from region of higher entropy to lower
entropy further out. If the motion is adiabatic and in pressure equilibrium, it
will decrease due to
\begin{equation}
  \label{eq:25}
  s(p, \rho) \propto \ln(p/\rho^{\gamma})
\end{equation}
and since it is less dense than the surroundings it will continue
to go up, thus run away, triggering the instability.

When people model the star, people first ignore convection, and model using
radiation, gravity, etc. Afterwards people add in convection and ask whether the
model is stable under convection.

The above reasoning hides an effect which is part of everyday life: difference
in chemical composition. If we put heavy fluid above light fluid, the heavy one
will sink and create fingers into the lighter one, creating the convection
instability.

The key thing to realize is that the pressure of an ideal gas depends on the
composition
\begin{equation}
  \label{eq:26}
  p = \sum_jn_jkT = \frac{\rho kT}{\mu m_p}
\end{equation}
where $\mu$ is the mean molecular weight, i.e.\ average mass per particle. $\mu
= 1/2$ for ionized H, $\mu = 4/3$ for He, and $\mu = 0.62$ for solar
metallicity. Now we can expand out the entropy derivative $ds/dz$
\begin{equation}
  \label{eq:27}
  \frac{ds}{dz} = \frac{d\ln p}{dz} - \gamma\frac{d\ln \rho}{dz} = \frac{d\ln T}{dz} - (\gamma - 1)\frac{d\ln \rho}{dz} - \frac{d\ln \mu}{dz}
\end{equation}
This means that $d\mu/dz > 0$ has a destabilizing effect, which confirms our
everyday experience: heavy on top of light is unstable. This is a continuous
version of the Rayleigh-Taylor instability.

This is very relevant in stars. The standard composition gradient in the normal
course of stellar evolution is that heavy elements show up towards the center,
and it has a stabilizing effect on the stellar structure.

Lets look at our assumption of $s$ being constant. Consider the diffusion time
scale
\begin{equation}
  \label{eq:28}
  t_\mathrm{diff} \sim H^2/\ell c \sim \tau H/c, \quad t_\mathrm{conv} \gtrsim H/c_s
\end{equation}
where $H$ is the characteristic scale height of system, and $\ell$ is the photon
free path. Therefore $t_\mathrm{diff} \lesssim t_\mathrm{conv}$ if $\tau
\lesssim c/c_{s}$. The place we see photon come from is where $\tau\sim 1$.
Somewhere between that and where $\tau \sim c/c_{s}$, the assumption of
adiabatic will fail, since we have rapid thermal diffusion in the surface layers
of the star.

Lets examine our reasoning again. When we thought of the fluid element going from
initial to final position, we assumed entropy being constant. However when we
have rapid diffusion, in the limit we have constant $T$ because rapid thermal
diffusion tries to even out the temperature gradient. This means at the final
position the temperature of the fluid element will be about the same as the
surroundings, and because it is still in pressure equilibrium, it will tend to
wipe out density gradients. This has a suppression effect on Buoyancy.

This effect has been seen in simulations, e.g.\
\href{http://arxiv.org/abs/1509.05417}{(Jiang et. al. 2015)}. In their
simulations the convective flux near the surface of the star is down by nearly
two orders of magnitude. The main difference in the simulation between the
surface of the star and interior is the optical depth.

Lets think about the microscopic energy transport due to photon transfer. This
effect of heat transport in normal stars because its free path is way
smaller than electron free path. However in degenerate plasmas like white dwarfs
and neutron stars, and in dilute and hot non-degenerate plasmas like solar
corona and hot accretion flows, thermal conduction can dominate. In the former
case thermal conduction is typically isotropic for lower magnetic fields like in
WDs, but anisotropic for NS. For the later case thermal conduction is highly
anisotropic because $l_{e}\gg \rho_{e}$. The main mechanism that determines free
path is Coulomb collision between electron and protons.

Anisotropic heat conduction can induce convection and fundamentally change the
way heat is conducted in the plasma. The setup we want to consider is similar to
the previous one. Consider an environment where the temperature is hot down
below and cold up above. A weak $B$ field points horizontally, which only
channels heat flow but weak enough to have now dynamical effect. We also assume
the thermal conduction time is much smaller than the buoyancy time. If we move a
blob of fluid from the hot region to the cold region along the $B$ field line.

Since thermal conduction is efficient, $T$ is maintained to be constant along
the magnetic field line. At the final position the blob has the same pressure as
the surroundings, with the same temperature as initially, which is hotter than
the surroundings, so it has less density than the surroundings. Therefore this
introduces a convective instability. The condition now becomes $dT/dz < 0$, and
the growth time is comparable to the dynamic time of the system.

An example simulation is covered in
\href{http://adsabs.harvard.edu/abs/2011MNRAS.413.1295M}{(McCourt et.\ al.\
  2011)}. The instability saturates by generating sustained convection and
amplifying the magnetic field.

There is sometimes a disconnect between actually doing the math of linear
instability and drawing the picture of the instability. So let us do some math
to illustrate the connection between the theory and the picture we discussed
above.

Consider the set of MHD including the Braginskii term (refer to Kunz's lecture).
The key term in the equations is the anisotropy term $\nabla\cdot \mathbf{Q}$,
where
\begin{equation}
  \label{eq:29}
  \mathbf{Q} = -\chi \hat{b} (\hat{b}\cdot\nabla)T
\end{equation}
which describes the heat flow along the magnetic field line. We've assumed that
the fluid element is always in pressure equilibrium with the surroundings. This
is actually an approximation, and what this means mathematically is that we keep
density perturbations due to buoyancy but not due to sound waves. Sound waves is
what we assume to wipe out the pressure differences. This also means
$\nabla\cdot \mathbf{v} = 0$.

This looks odd at first since it says incompressible fluid, but we are
interested in density variations. This is simply done to get sound waves out of
the problem, and only keeping the density perturbations due to gravity.

Lets assume the perturbations vary as $e^{i(\mathbf{k}\cdot \mathbf{x} - \omega
  t)}$, where $\mathbf{k}$ is in the $x$ direction, and $B$ field is in the $x$
direction as well. If we perturb the heat flux (Eulerian, at fixed place)
\begin{equation}
  \label{eq:30}
  \begin{split}
    \delta Q &= -\chi \delta \hat{b}(\hat{b}\cdot\nabla) T \\
    &-\chi \hat{b}(\delta \hat{b}\cdot\nabla) T \\
    &-\chi \hat{b}(\hat{b}\cdot\nabla) \delta T
  \end{split}
\end{equation}
The first term is zero due to our assumption that $T$ is constant along the $B$
field direction. Due to our setup there is no first order change in the
magnitude of the $B$ field, so $\delta \hat{b} = \delta \mathbf{B} / B_{0}$. We also
have
\begin{equation}
  \label{eq:31}
  \begin{split}
    \frac{\partial \mathbf{B}}{\partial t} &= \nabla\times (\mathbf{v}\times \mathbf{B}) \\
    &= (\mathbf{B}\cdot \nabla)\mathbf{v} - (\mathbf{v}\cdot\nabla)\mathbf{B}
  \end{split}
\end{equation}
the second term is zero, so we have the linearized version as
\begin{equation}
  \label{eq:32}
  -i\omega \delta \mathbf{B} = ik B_0\delta \mathbf{v} = ikB_0(-i\omega)\delta \boldsymbol{\xi}
\end{equation}
so we can write $\delta \hat{b} = ik\delta\boldsymbol{\xi}$. Plugging this into
our equation of heat flux we get
\begin{equation}
  \label{eq:33}
  \begin{split}
    \delta Q &= -\chi (ik\boldsymbol{\xi}\cdot\nabla)T \hat{b} - \chi ik \hat{b} \delta T \\
    &= -ik\chi (\xi_z\frac{\partial T}{\partial z} + \delta T)\hat{b}
  \end{split}
\end{equation}
The divergence of this is simply bringing in another $ik$, so we have
\begin{equation}
  \label{eq:34}
  \nabla\cdot\delta Q = k^2\chi(\xi_z\frac{\partial T}{\partial z} + \delta T)
\end{equation}
Since the time scale for thermal conduction is very short, which means
\begin{equation}
  \label{eq:35}
  t_\mathrm{cond} \sim \frac{\lambda^2}{\chi} = \frac{1}{k^2\chi} \to 0
\end{equation}
If we take this limit, our equation will blow up, which is bad, unless we also
take the terms inside the brackets to be zero. What this means is that rapid
conduction implies that
\begin{equation}
  \label{eq:36}
  \delta T + \xi_{z}\frac{\partial T}{\partial z} = 0 = \Delta T
\end{equation}
which is exactly the Lagrangian change in temperature. Which means that moving
with the blob, the temperature doesn't change.

Lastly if we now use pressure equilibrium, which says
\begin{equation}
  \label{eq:37}
  \frac{\delta\rho}{\rho} + \frac{\delta T}{T} = 0
\end{equation}
therefore we can replace $\delta T$:
\begin{equation}
  \label{eq:38}
  \frac{\delta \rho}{\rho} = \xi_z\frac{d\ln T}{dz}
\end{equation}
What this means is that if $\xi_z$ is positive and if $dT/dz$ is negative, then
$\rho$ will decrease and it corresponds to a buoyancy system.

One thing that we didn't get is the growth rate of the instability. What we did
was deriving the condition of the instability. To get the growth rate, we need
to keep all the linear terms and do the whole perturbation theory. One will
eventually get a growth rate
\begin{equation}
  \label{eq:39}
  \gamma \approx \left[ g \left| \frac{d\ln T}{dz} \right| \right]^{1/2}
\end{equation}

When the magnetic field is too strong, the magnetic force will prevent the
bending of magnetic field lines, so it will suppress the instability. This
happens roughly when $\beta \sim 1$.

Is this an overly simplified problem because we have only horizontal magnetic
field? It turns out it is more generic than our simple setup. Even when there is
vertical magnetic field which allows temperature exchange vertically, it doesn't
wipe out the instability.

If we revert the temperature gradient, then this argument tells us that this
system is stable, but it is wrong. The physics is different, because this
instability only happens when there is a component of the magnetic field in the
vertical direction. This is called the heat flux-driven buoyancy instability
(HBI). This instability operates best when $\mathbf{k}$ is 45 degrees with the
$\mathbf{B}$ field. We can derive this using the same algebra as above, which is
worked out in the notes.

It turns out that a weakly magnetized plasma with anisotropic heat transport is
always buoyantly unstable. This manifests mostly in hot plasma in galaxy
clusters where both thermal conduction and viscosity are important.

\end{document}
