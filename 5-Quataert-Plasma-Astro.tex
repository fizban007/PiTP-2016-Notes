\documentclass[letterpaper, 11pt]{article}

\usepackage[pdftex]{graphicx}
\usepackage{epstopdf}
\DeclareGraphicsRule{*}{mps}{*}{} 

\usepackage{amsmath, amsthm, amssymb}
\usepackage{listings}
\usepackage{float}
\usepackage{enumerate}
% \usepackage{mystyle}
\usepackage{hyperref}
\usepackage{tikz}
\usepackage{fancyheadings}
\usepackage{tensor}
\usepackage{mathrsfs}
\usetikzlibrary{positioning}
\usetikzlibrary{decorations.pathmorphing}
\usetikzlibrary{arrows}
\usetikzlibrary{decorations.markings}
%\usepackage{fullpage}
\usepackage[left=0.75in, top=1.25in, right=0.75in, bottom=1.25in]{geometry}
\newcommand{\lambdabar}{{\mkern0.75mu\mathchar '26\mkern -9.75mu\lambda}}
\newcommand{\Jmath}{J}

\numberwithin{equation}{section}
\numberwithin{figure}{section}

\begin{document}

\title{Selected Topics in Plasma Astrophysics}
\author{Eliot Quataert}
\date{July 19, 2016}

\maketitle

\section{Lecture 1}

\subsection{Astrophysical Plasmas in General}

Plasma astrophysics is a very broad subject, encompassing a lot of different sub
fields. There are a lot of different techniques to study them as well. For
relativistic plasma physics, there are force-free electrodynamics to study e.g.\
pulsar magnetospheres. There is a family of (GR)(M)HD that study accretion,
jets, etc. There is PIC that is used to study shocks, and Dynamic space-time +
MHD to study compact object mergers.

For non-relativistic theory we use force-free for e.g.\ solar corona, MHD for
e.g.\ star formation, disks, etc., and kinetic theory for shocks, reconnection,
turbulence, etc. Even within MHD, there is a wide range of fluid models which
are useful for different situations.

In the first lecture we focus on fluid models with some kinetic elements.

\subsection{Stellar Winds}

Stellar winds is a topic that is rich and has influence over many other branches
of plasma astrophysics, and it will be our topic today. We will be discussing
different kinds of winds:
\begin{itemize}
    \item Thermally driven winds (from sun-like stars): hydrodynamic theory, kinetic
  theory
    \item Magnetocentrifugically driven winds: MHD, rotation as energy source
      but trapped by $B$ fields.
    \item Radiation pressure driven outflow: $L > L_\mathrm{edd}$: continuum
      driven or line-driven, depending on the spectrum
\end{itemize}

\subsection{Solar Corona and Wind}

Solar corona and outflow is an important part of solar dynamics. The solar
outflow is very small in terms of mass and energy. However the solar wind is
efficiently extracting the angular momentum from the sun. Its spin has slowed
down by a factor of 30 to 50 since its birth. The time scale of $dJ/dt$ is on
the order of $10^{10}$ years.

The solar corona is not in thermal equilibrium. We know that $T_i \gg T_p \geq
T_{e}$, and is anisotropic $T_{\perp}\geq T_{\parallel}$. The corona is mostly
collisionless where $\ell_{mfp}\sim 10^{8}\rho_\mathrm{Larmor}$. It turns out
that despite the collisionlessness we can use fluid theory to somewhat describe
the solar wind.

Lets start by talking about the Parker model. Consider a time-independent
outflow, the continuity equation is
\begin{equation}
  \label{eq:1}
  \nabla\cdot (\rho \mathbf{v}) = 0 \Longrightarrow r^2\rho v_r = \text{const}
\end{equation}
From this the mass loss rate is simply $\dot{M} = 4\pi r^2\rho v_r$ which is
also a constant. The radial velocity equation reads
\begin{equation}
  \label{eq:2}
  v\frac{dv}{dr} = -\frac{1}{\rho}\frac{dP}{dr} - \frac{GM}{r^2}
\end{equation}
If we assume temperature being constant, pressure $P = \rho c_s^2 = \rho
kT/m_p$, then we have this equation
\begin{equation}
  \label{eq:3}
  \frac{1}{v_r}\frac{dv_r}{dr}(v_r^2 - c_s^2) = \frac{2c_s^2}{r} - \frac{GM}{r^2}
\end{equation}
This equation describes spherical wind, or spherically symmetric (Bondi)
accretion.

There is a special point in this equation called the sonic point. This is where
$v_r = c_s$. This requires that the right hand side is also zero
\begin{equation}
  \label{eq:4}
  \frac{2c_s^2}{r} = \frac{GM}{r^2}
\end{equation}

In the regime where $v < c_s$, we can ignore the left hand side of equation
\eqref{eq:3}. This gives a spatial distribution of density. However when this is
true, the pressure is very big. In fact if we require that the flow goes through
the sonic point peacefully, we can uniquely obtain a unique outflow solution
from this equation. TODO: clean this paragraph up

\subsection{Magnetic Field}

Around the sun there are magnetic field lines that open up to infinity, and
field lines that close back to the star itself. For a non-rotating star, the
above theory can be thought of as describing the acceleration of the outflow
along the open magnetic field lines. The only non-trivial thing is the structure
of the global magnetic field.

However now let's consider when the star is rotating. Lets simplify the actual
magnetic field to a split monopole configuration, which can be called a
``theorist's monopole''. This is a useful toy model where we can calculate
structures of the outflow relatively easily. If we have a very powerful outflow
that blows up the magnetic field lines, then at large distances it will actually
look like a split monopole field.

We will now focus on the equatorial plane. Imagine the magnetic field lines have
infinite tension. The outflow will rigidly follow the field lines, and forced to
corotate with the star. As they flow out their rotation velocity will increase
with radius, as well as their specific angular momentum. Therefore they extract
angular momentum from the star. The magnetic field acts as a medium that
extracts angular momentum from the star and transfer it to the outflow.

The corotation only can keep up till some point, where the magnetic energy
becomes comparable to the kinetic energy of the material
\begin{equation}
  \label{eq:5}
  \frac{B^2}{8\pi} \sim \rho v_r^2
\end{equation}
At this point there is so much energy in the gas that you can't treat the
magnetic field as rigid. This is called the \emph{Alfven point} $r_A$. This is also
when
\begin{equation}
  \label{eq:6}
  v_r = v_A \sim \frac{B}{\sqrt{4\pi\rho}}
\end{equation}

Lets look at this outflow in more detail. Let
\begin{equation}
  \label{eq:7}
  \mathbf{v} = v_r\hat{r} + v_{\phi}\hat{\phi},\quad \mathbf{B} = B_r\hat{r} + B_{\phi}\hat{\phi}
\end{equation}
We want to use the conservation of momentum and energy again like before to
write down the equation governing the outflow. We have
\begin{align}
  \dot{M} = 4\pi r^2 \rho v_r = \text{const}, \quad \frac{\partial \mathbf{B}}{\partial t} = \nabla\times(\nabla \times \mathbf{B}) = 0
\end{align}
We can show that
\begin{equation}
  \label{eq:8}
  \frac{B_{\phi}}{B_{r}} = \frac{v_{\phi} - r\Omega}{v_r}
\end{equation}
The way to think about this result is that, if there is perfect corotation, then
there is no $B_{\phi}$. If $v_{\phi}$ can't keep up with $r\Omega$ the magnetic
field will start to lag behind and be swept back, leading to a negative
component of $B_{\phi}$.

The next equation which is important is the momentum equation in the $\phi$
direction. It tells us how does angular momentum change
\begin{equation}
  \label{eq:9}
  \rho v_r\frac{d}{dr}(rv_{\phi}) = \frac{B_r}{4\pi}\frac{d}{dr}(rB_\phi)
\end{equation}
If we multiply by $r^2$ we can easily integrate this equation, giving
\begin{equation}
  \label{eq:10}
  L = rv_{\phi} - \frac{rB_{\phi}B_r}{4\pi\rho v_r} = \text{const}
\end{equation}
There are two contributions to the angular momentum: $L_\mathrm{gas}$ and
$L_\mathrm{mag}$. Therefore the angular momentum is carried out both by the gas
and the magnetic field, and the total angular momentum is a constant in the
problem.

To find what the constant is, we can substitute equation \eqref{eq:8} into this
equation and solve:
\begin{equation}
  \label{eq:11}
  v_{\phi} = r\Omega \left( \frac{v_r^2L/r^2\Omega - v_{Ar}^2}{v_r^2 - V_{Ar}} \right)
\end{equation}
where $v_{Ar}^2 = B_r^2/4\pi\rho$. This has the same structure as before: the
denominator is zero at some point. Here is apparent that the magnetic field
component that determines the Alfven point is the radial point. Therefore at the
Alfven point we have
\begin{equation}
  \label{eq:12}
  L = r_A^2\Omega
\end{equation}
This means that the wind solution determines exactly how much angular momentum
is extracted from the star. This idea is also why the outflow model is so
important in plasma astrophysics. Because the magnetic field can be very strong,
we could have $r_A \gg R_{*}$. The rate of angular momentum loss is
\begin{equation}
  \label{eq:13}
  \dot{\Jmath} = \dot{M} L = \dot{M}r_A^2\Omega
\end{equation}
so the time scale is
\begin{equation}
  \label{eq:14}
  t = \frac{J}{\dot{\Jmath}} \sim \frac{MR_*^2}{\dot{M}r_{A}^2}
\end{equation}
Due to the large ``level arm'' of the Alfven radius, this time scale can be much
smaller than the time scale of mass loss. This efficient extraction of angular
momentum due to magnetic field will occur in other places in plasma astrophysics
as well.

The last thing we want to talk about is the energy carried by the outflow.
Consider hydrodynamics, we have the Bernoulli constant which is essentially
energy per unit mass:
\begin{equation}
  \label{eq:15}
  B_l = \frac{1}{2}v^2 + \phi + h,\quad h = \frac{5}{2}\frac{kT}{m_p}
\end{equation}
where $h$ is called the enthalpy $h = \gamma/(\gamma - 1) kT/m$. In thermally
driven wind, if we heat up the gas it will be able to escape. However in MHD we
have an additional loss of energy in the form of Poynting flux, which also taps
into the rotational energy of the star. We find that
\begin{equation}
  \label{eq:16}
  B_l + \frac{S_r}{\rho v_r} = \text{const}
\end{equation}
where $S_r$ is the radial Poynting flux. Given the magnetic field structure it
is easy to calculate the Poynting flux
\begin{equation}
  \label{eq:17}
  \frac{S_r}{\rho v_{r}} = -\frac{r\Omega B_rB_{\phi}}{4\pi \rho v_{r}}
\end{equation}
If we call this constant $\epsilon$, then it is like the angular momentum flux,
it has a gas component and also a magnetic field component, where in fact
$\epsilon_\mathrm{mag} = \Omega L_\mathrm{mag}$.

Now the question is which of these is more important. Lets roughly estimate it
to get a feeling:
\begin{equation}
  \label{eq:18}
  \epsilon_\mathrm{gas} \sim \frac{kT_\mathrm{corona}}{m_p},\quad \epsilon_\mathrm{mag} \sim r_A^2\Omega^2
\end{equation}
For the sun, the gas energy flux is much larger than the magnetic energy flux,
and most of the energy outflow is thermal energy. This is a thermally driven
wind. In the other regime most of the energy that goes to infinity is the
rotational energy of the central object. This is what we call a
magnetically driven outflow. Which regime we have depends no the
temperature of the corona, rate of rotation, and Alfven radius.

Lets come back to the question is why do we use a fluid model at all for the
sun. The sun is a slowly rotating object. Around the sun the presence of
magnetic field means $\rho_\mathrm{Larmor} \ll R$, so plasma streams only in one
direction. Along the $B$ field pressure is the origin of acceleration, but fluid
theory is okay within order of about a few. Kinetic instabilities also helps us
by limiting how much our distribution function can deviate from Maxwellian.

The state of the art of solar corona study is to understand how particles are
heated in the solar corona. This heating has been measured from the corona all
the way to much further away from the Earth. The temperature profile is
shallower than adiabatic, so there must be heating going on. One idea is that
Alfven waves are launched and the turbulence keep heating the plasma.

The time-reversal of this outflow wind model is the spherical Bondi accretion.
There is also the theory of magnetically directed accretion to the central object
analogous to the magnetically driven wind models. But it works not as good
because in the wind model the outflow is driven by the rotation of the star, and
same can't be said for accretion models.

\subsection{Radiative Driven Winds}

Lets now briefly talk about radiation pressure driven winds. Around RGB and AGB
stars, a lot of dust forms in stellar atmosphere which have high $\kappa$. They
feel the radiation pressure much higher than the gas but are coupled to the gas
through collision, therefore driving a wind. Around massive stars $L >
L_\mathrm{edd}$ on metal lines, and the plasma absorption of metal line photons
drives the wind.

In thermally driven winds the energy outflow is $\dot{M}$ times the speed of
sound squared. However for the radiation pressure driven winds we should think
of momentum flux due to photons
\begin{equation}
  \label{eq:19}
  \dot{P} \sim \dot{M}v_{\infty} \sim L/c,\quad v_{\infty} \sim v_\mathrm{escape}
\end{equation}
We can derive this from the momentum equation directly
\begin{equation}
  \label{eq:20}
  v\frac{dv}{dr} = -\frac{1}{\rho}\frac{dP}{dr} - \frac{GM}{r^2} + \kappa \frac{F}{c}
\end{equation}
where the right hand side has a new term which is an average radiation pressure
force over all wavelengths. If we only focus on the new term, we can write
\begin{equation}
  \label{eq:21}
  v\frac{dv}{dr} = \frac{\kappa L}{4\pi r^2 c}
\end{equation}
Multiplying both sides by $4\pi r^{2}\rho$ we have
\begin{equation}
  \label{eq:22}
  \dot{M}\frac{dv}{dr} = \rho\kappa\frac{L}{c}
\end{equation}
Integrating this over all radii to infinity we now have
\begin{equation}
  \label{eq:23}
  \dot{M}v_{\infty} = \tau\frac{L}{c}
\end{equation}
where $\tau$ is the optical depth $\tau = \int \rho\kappa\,dr$. That is exactly
we have, and the optical depth is the one uncertain numerical coefficient.

For line-driven winds refer to (Lucy \& Solomon 1970; Castor, Abott, Klein 1975).

For applications of this wind theory in other models/fields, refer to the
original slides! Mentioned models are thermally driven galactic winds, line
driven winds from accreting black holes, magnetized winds from accretion disks.

\section{Lecture 2}

\subsection{Instabilities in Fluids and Plasmas}


\end{document}
