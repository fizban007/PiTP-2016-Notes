\documentclass[letterpaper, 11pt]{article}

\usepackage[pdftex]{graphicx}
\usepackage{epstopdf}
\DeclareGraphicsRule{*}{mps}{*}{} 

\usepackage{amsmath, amsthm, amssymb}
\usepackage{listings}
\usepackage{float}
\usepackage{enumerate}
% \usepackage{mystyle}
\usepackage{hyperref}
\usepackage{tikz}
\usepackage{fancyheadings}
\usepackage{tensor}
\usepackage{mathrsfs}
\usetikzlibrary{positioning}
\usetikzlibrary{decorations.pathmorphing}
\usetikzlibrary{arrows}
\usetikzlibrary{decorations.markings}
%\usepackage{fullpage}
\usepackage[left=0.75in, top=1.25in, right=0.75in, bottom=1.25in]{geometry}
\newcommand{\lambdabar}{{\mkern0.75mu\mathchar '26\mkern -9.75mu\lambda}}
\newcommand{\Jmath}{J}

\numberwithin{equation}{section}
\numberwithin{figure}{section}

\begin{document}

\title{Relativistic Astrophysics}
\author{Charles Gammie}
\date{July 25, 2016}

\maketitle

\section{Lecture 1 --- Accretion Disks}

\subsection{Child's Garden of Astrophysical Disks}

Here is a (incomplete) list of astrophysical disks
\begin{itemize}
\item Galactic disk: spiral/elliptical
\item Supermassive BH: quasar/Seyfert/LINER/LLAGN/TDE
\item Stellar mass BH: microquasar/GRB
\item Neutron star: LMXB/HMXB/GRB
\item White dwarf: dwarf nova/nova
\item Protostar: protoplanatory/debris
\item Planet: protolunar disk/planetary rings
\end{itemize}

Consider the galaxy NGC 4258. If we zoom in to $~0.1\,\mathrm{pc}$, we see maser
spots around the central engine. The velocities of these spots fall nicely on a
Keplerian orbit. The combination of the acceleration, velocity, and angular
position of these maser spots allows us to measure to mass of the central
object.

Here let us introduce the first dimensionless parameter of a thin disk
\begin{equation}
  \label{eq:1}
  \frac{H}{R}\ll 1, \quad H = \frac{c_s}{\Omega} \propto T^{1/2}
\end{equation}
For this system this parameter is $\sim 10^{-3}$. This is an example of a thin
disk. The disk of NGC 4258 is not a flat disk, since one can see warps following
the maser spots. The disk is actually made of rings at the positions of the
masers.

This particular disk is relatively far away from the object, and the disk is
heated due to the radiation from the central object.

The next example is the center of our galaxy. We know that the accretion flow
near $\sim 10 r_{g}$ emits synchrotron radiation which is the target of the
Event Horizon Telescope (EHT). The mean free path of Coulomb scattering in Sgr
A* is very large, so the plasma is almost collisionless. This allows us to
introduce the second number
\begin{equation}
  \label{eq:2}
\kappa_n = \frac{\lambda_{mfp}}{R}
\end{equation}
which is about $10^5$ for this system.

The next example is the HL Tau. This is a cold disk of protoplanetary disk. At
$50\,\mathrm{au}$ the temperature is less than $100\,K$. Here we introduce the
third number which is the magnetic Reynolds number
\begin{equation}
  \label{eq:3}
  \mathrm{Re}_M = \frac{c_s H}{\eta}
\end{equation}
In regions of this system that we can measure with ALMA, this number is less
than 1.

The last number we introduce is
\begin{equation}
  \label{eq:4}
  Q = \frac{c_s\Omega}{\pi G \Sigma}
\end{equation}
where $\Sigma$ is the surface density of the disk. When this number is less than
one, self gravity becomes important.

The last example is the moon. When people look at the moon most people do not
see a disk. The theory of the formation of the moon involves an impact on earth,
throwing some mass from the earth which forms a disk around it. This is a system
where the magnetic Reynolds number might play a role.

One more system we want to mention is the SS Cyg which is a Cataclysmic
variable, formed by a massive star overflowing its Roche lobe, feeding a disk
around its companion which is a WD. We have observed this system since 1896 and
it alternates between a quiescent state and an active state. This system may
have a low magnetic Reynolds number at quiescent state, but has a high magnetic
turbulence during its outbursts.

\subsection{Disk Evolution}

Disks are a class of special objects in astrophysics. The angular momentum is
conserved, but its kinetic energy is easily converted to thermal energy due to
various instabilities in the disk, and eventually radiated away. Therefore
understanding the disk evolution is all about understanding the evolution of
angular momentum.

In a thin disk, dynamical equilibrium is reached in a dynamic timescale $\Delta
t \sim \Omega^{-1}$, where
\begin{equation}
  \label{eq:5}
  \Omega = \left( \frac{GM}{R^3} \right)1/2 + O \left( \frac{H}{R} \right)^2
\end{equation}
However it is possible for there to be long living excitations which live in
circular orbits, in addition to tilts and warps in the disk.

Thermal equilibrium is reached when $Q^+ \sim Q^{-}$, when heating rate balances
the cooling rate. The time scale is
\begin{equation}
  \label{eq:6}
  \Delta t \sim \Sigma c_s^2 / Q^+ \sim (\alpha\Omega)^{-1}
\end{equation}
where the $\alpha$ parameter is very important and it describes the intensity of
turbulence in the disk, and the above equation can be taken as the definition.
It relates the thermal time scale to the dynamical time scale. Its value,
usually at $10^{-2}$ means that the thermal time scale is much longer than the
dynamical time scale.

The last time scale is the inflow equilibrium $\dot{M} \sim \mathrm{const}$. The
time scale is
\begin{equation}
  \label{eq:7}
  \Delta t \sim \frac{M_\mathrm{disk}}{\dot{M}} \sim (\alpha\Omega)^{-1}\left( \frac{R}{H} \right)^2
\end{equation}
which is again longer by a factor of the scale height squared.

Lets write down the disk evolution equation
\begin{equation}
  \label{eq:8}
  \partial_t\Sigma = \frac{2}{r}\partial_r \left( \frac{\Omega}{r\kappa^2}\partial_r(r^2W_{r\phi}) - \frac{\Omega}{\kappa^2}\tau \right) + \dot{\Sigma}_\mathrm{ext}
\end{equation}
where $\Sigma$ is the surface density, $\Omega$ the orbital frequency, $\kappa$
the epicyclic frequency which is similar to $\Omega$, $W_{r\phi}$ is the shear
stress, $\tau$ is connected to external torques per area, and finally
$\dot{\Sigma}_\mathrm{ext}$ is the mass gain/loss from infall or from wind loss.
The full derivation of this equation is left as an exercise. The hint is to
start from the angular momentum conservation equation together with conservation
of mass. Another hint is that
\begin{equation}
  \label{eq:9}
  \frac{d}{dr}j = \frac{r\kappa^2}{2\Omega}
\end{equation}
where $j$ is the specific angular momentum.

The standard model for disks is the $\alpha$ disk model. It was introduced by
(Shakura \& Sunyaev 1973) and (Linden-Bell \& Pringle 1974). It adopts simple
scaling argument for diffusion of angular momentum by turbulence, by just
modeling the turbulent viscosity
\begin{equation}
  \label{eq:10}
  \nu \sim \alpha c_s H
\end{equation}
In classical $\alpha$ disk models one ignores external torques, infall/winds,
etc. Thus one throws away the last two terms in the disk equation, with only the
term with $W_{r\phi}$ surviving.

In the $\alpha$ disk model we assume the disk is thin and Keplerian $\Omega
\approx (GM/r^3)^{1/2}$. We also assume vertical equilibrium $H \sim
c_s/\Omega$. The opacity is assumed to be related to temperature and density $\kappa \sim
\kappa_0\rho^aT^{b}$. We have a prescription of vertical integration of the
density $\Sigma \sim 2\rho H$, which is a crude integral. We also estimate the
optical depth to be $\tau \sim \Sigma \kappa / 2$. The surface temperature is
determined by the fact that heating by viscous processes $F = \sigma
T_\mathrm{eff}^4 \sim (9/8)\Sigma \nu \Omega^2$, where $T_\mathrm{eff}^4 \sim
\sigma T^4/\tau$. Finally in the steady state $\dot{M} = 3\pi \Sigma \nu$.

As an example, consider a steady state disk around a stellar mass BH. In the
inner zone, radiation pressure is much larger than the gas pressure, and
electron scattering dominates the opacity. We can find that
\begin{align}
  T &\sim 4.3\times 10^7 \alpha^{-1/4}m^{-1/4}x^{-3/8}\,K \\
  \Sigma &\sim 0.4 x^{3/2}\alpha^{-1}\dot{m}^{-1}\,\mathrm{g/cm^2} \\
  H/r &\sim 10 \dot{m}x^{-1}
\end{align}

Now lets come back to the disk evolution equation and ask what are $W_{r\phi}$,
$\tau$, and $\Sigma_\mathrm{ext}$. Part of the answer only lies in the global
analysis of turbulence in the disks.

\subsection{Turbulence in Disks}

In $\alpha$ disk model the turbulent diffusion of angular momentum is simply
postulated. Possible sources of turbulence include magnetorotational instability
(MRI), gravitational instability, zombie vortex instability, subcritical
baroclinic instability, and vertical shear instability. The last three are
related to instabilities in fluids. The zombie vortex instability is generated
due to vortices generated in the disk sharing angular momentum through sound
waves. The subcritical baroclinic instability is loosely related to the
baroclinic instability that drives weather. The vertical shear instability is
when shearing motion is faster than orbital motion and driving the flow of
angular momentum at sound speed.

We will focus on the MRI instability (Balbus \& Hawley 1991). This is due to
local, linear instability of weakly magnetized disks driven by exchange of
angular momentum. The condition of the instability does not involve boundary
conditions and occur in a localized region of the disk, and can be treated using
the WKB approximation.

We will start with a simple mechanical analogy where 2 masses connected by a
spring are in orbit around a central body. The orbital frequency is given by
Keplerian motion $\Omega^2 = GM/R^{3}$. The masses can be thought as fluid
elements, and spring can be a model of magnetic field line tension which
provides restoring force. We denote the frequency of the spring as $\gamma$.

Lets setup $x$-$y$ coordinates for the orbital motion where $x$ is in radial
direction, and $y$ points opposite to the orbital motion. We know that
\begin{equation}
  \label{eq:11}
  \ddot{x} = -2\Omega \dot{y} + 3\Omega^{2}x - \gamma^2x
\end{equation}
The first term is the Coriolis force, and the second combines the centrifugal
force and the gravitational force in an expansion, and the third term is the
spring force. The second equation is
\begin{equation}
  \label{eq:12}
  \ddot{y} = 2\Omega \dot{x} - \gamma^{2}y
\end{equation}

As a first go, lets set $\gamma = 0$ and assume $x$ and $y$ scale with
$e^{-i\omega t}$, the equations become
\begin{align}
  \label{eq:13}
  -\omega^2x &= -2\Omega(-i\omega) y + 3\Omega^2 x \\
  -\omega^2y &= 2\Omega(-i\omega)x
\end{align}
This simply gives $\omega^2 = \Omega^2$ which means that the natural modes
oscillate at the orbital frequency.

Now lets add the $\gamma$ term, and we get
\begin{align}
  \label{eq:14}
  -\omega^2x &= -2\Omega(-i\omega) y + 3\Omega^2 x - \gamma^2x \\
  -\omega^2y &= 2\Omega(-i\omega)x - \gamma^2y
\end{align}
And solving the dispersion relation gives
\begin{equation}
  \label{eq:14}
  \omega^4 - \omega^2(\Omega^2 + 2\gamma^2) + \gamma^2(\gamma^2 - 3\Omega^2) = 0
\end{equation}
Notice that there is a zero solution condition when $\gamma^2 = 3\Omega^2$, and
it is a transition from positive to negative frequency square. If we plot
$\omega^2/\Omega^2$ vs $\gamma^2/\Omega^2$, there are two sets of roots. One is
simply spring motion. The other starts with $\omega^{2} = 0$, and between
$\gamma^2 = 0$ and $3\Omega^2$ we have $\omega^2 < 0$ where there is an
instability. The fastest growing mode of the instability is at
$\gamma^2/\Omega^2 = 15/16$ and $\omega^2/\Omega^2 = -9/16$.

This is directly analogous to an MHD problem. Lets zoom in on the disk and
assume it is penetrated by a weak vertical magnetic field. This instability is
related to the magnetic instability where the field is bent, and the fluid
element above and below are coupled by Alfven waves. In this case we have
\begin{equation}
  \label{eq:15}
  \gamma^2 \to (\mathbf{k}\cdot \mathbf{v}_A)^2
\end{equation}

Lets list some facts about MRI linear theory. The ideal fluid instability only
requires $d\Omega^2/dr < 0$. The maximum growth rate is $(3/4)\Omega$ for
Keplerian motion. The fastest growing mode is given above. We know that there is
a local instability when there is vertical field, but a local instability also
occurs even when there is only toroidal field.

However linear theory doesn't tell us what happens in the regime when linear
growth saturates. To study this regime we need simulations. In simulations
people have used local or global setups. There are models with explicit small
dissipation terms, or ILES which self-consistently evolves turbulence cascade
from the scale where energy is injected down to the scale of dissipation. There
are also isothermal models with equation of state $P = c_s^2\rho$, or one can
use energetically self-consistent equation of state.

Some simulations were shown\dots

Simulations of MRI told us a few things. In 2D MRI leads to turbulence and
$\alpha$. We know that in 2D MRI does not converge, so $\alpha$ depends on
resolution. In 3D MRI also leads to turbulence and $\alpha$, but sometimes it
does converge. We know that it converges when explicit dissipation is used.
There are issues with ILES models with unstratified shearing boxes. We also
learned that $\alpha$ depends on many parameters. It depends on the local height
$z$, and it increases away from the central plane. It also depends on the
magnitude of the vertical magnetic field that threads the disk $\left<
  B_z\right>$. When the vertical magnetic field increases the viscosity also
increases. It depends on the magnetic Reynolds number and $\mathrm{Pr}_{M} =
\nu/\eta$.

\subsection{Current Problems in Disk Theory}

Ran out of time\dots

\end{document}
