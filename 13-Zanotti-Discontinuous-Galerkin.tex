\documentclass[letterpaper, 11pt]{article}

\usepackage[pdftex]{graphicx}
\usepackage{epstopdf}
\DeclareGraphicsRule{*}{mps}{*}{} 

\usepackage{amsmath, amsthm, amssymb}
\usepackage{listings}
\usepackage{float}
\usepackage{enumerate}
% \usepackage{mystyle}
\usepackage{hyperref}
\usepackage{tikz}
\usepackage{fancyheadings}
\usepackage{tensor}
\usepackage{mathrsfs}
\usetikzlibrary{positioning}
\usetikzlibrary{decorations.pathmorphing}
\usetikzlibrary{arrows}
\usetikzlibrary{decorations.markings}
%\usepackage{fullpage}
\usepackage[left=0.75in, top=1.25in, right=0.75in, bottom=1.25in]{geometry}
\newcommand{\lambdabar}{{\mkern0.75mu\mathchar '26\mkern -9.75mu\lambda}}
\newcommand{\Jmath}{J}

\numberwithin{equation}{section}
\numberwithin{figure}{section}

\begin{document}

\title{Discontinuous Galerkin Methods for Hyperbolic PDEs}
\author{Olindo Zanotti}
\date{July 26, 2016}

\maketitle

\section{Lecture 1}

\subsection{Basic Terminology}

Lets refresh some terminologies. Finite difference methods evolve in time the
point-values of the solution. Finite volume methods evolve in time the cell
averages of the solution, which may allow one to e.g.\ go to unstructured
meshes. Discontinuous Galerkin methods evolve in time the so-called degrees of
freedom of the solution, i.e.\ the expansion coefficients with respect to given
basis functions
\begin{equation}
  \label{eq:1}
  U_h(x, t) = \sum_{l=0}^M\psi_l(x) \hat{U}^n_l(t)
\end{equation}

\subsection{Recap about numerical methods for hyperbolic PDEs}

Lets assume a system of PDEs in conservative form
\begin{equation}
  \label{eq:2}
  \frac{\partial U}{\partial t} + \nabla\cdot F(U) = S
\end{equation}
where $U$ is the vector of so-called conserved quantities, and $F(U)$ is a
non-linear flux tensor which depends on the state $U$. It can usually be written
as
\begin{equation}
  \label{eq:3}
  \partial_tU + A\cdot \nabla U = S
\end{equation}

For the first form, the system of PDEs can be solved through conservative
numerical schemes. The second form requires different approaches, like
path-conservative numerical schemes.

Hyperbolic PDEs are everywhere in physical sciences. One can even argue that all
physical systems are hyperbolic. Examples are hydro, MHD, GR, etc.

Lets refresh our memory about the finite volume method. We integrate the
hyperbolic equation across the cell, then integrate over the time. Then we use
the volume averages to replace the integrals and write down the evolution of the
averages. To evolve these averages, we evolve a Riemann problem at every
contact point between the cells. We don't usually analytically solve Riemann
problems at every interface, but approximate them using e.g.\ characteristics of
the PDE system.

A numerical scheme is called a conservative scheme if it is based on the
conservative form of the PDEs. More specifically if the numerical flux depends
on the values taken by $U$ on the neighboring cells, and the flux function
$\mathcal{F}$ reduces to the true physical flux int he case of constant flow
(just a consistency check).

There are two theorems about conservative numerical schemes. Conservative
numerical schemes, if convergent, converge to the weak solution of the problem
(Lax \& Wendroff 1960). Non-conservative numerical schemes don't\dots

About weak solutions. A function $U$ is called a weak solution of the
conservative equation if it satisfies the so-called weak formulation for all
functions $\psi$ which are continuously differentiable functions with compact
support.

In Godunov's first order method the left and right sides of the Riemann problem
are the constant cell averages. To go beyond the first order, lets define the
local error $E_{j}$ which is the difference between the exact and numerical
solutions at $x_j^n$. The order of the accuracy of the scheme can be defined as
$p_j = \mathrm{min}(m_j, n_j)$ where
\begin{equation}
  \label{eq:4}
  E_j = C_1\Delta x^{m_j} + C_2\Delta t^{n_j}
\end{equation}
However when there is a discontinuity the error is always 1st order.

A theorem by Godunov says that A linear (i.e.\ with constant coefficients) and
monotonicity-preserving scheme is at most first-order accurate.
Monotonicity-preserving means that if $U_j^n \geq U_{j+1}^n$ implies that
$U_j^{n+1} \geq U_{j+1}^n$. This means that to go to higher order we need to
sacrifice one of them, and we sacrifice linearity.

For example we have the Total Variation Diminishing (TVD) schemes where we
provide a piecewise linear reconstruction of $U_{j}(x) = U_j + \sigma_j(x -
x_j)$ inside each cell. The nonlinearity is hidden in the constant slope
$\sigma_j$, e.g.\ the minimod slope limiter. Other non-linear schemes can also
be constructed.

Up to the second order, finite difference schemes and finite volume schemes are
essentially the same.

One more method is the method of lines. We can regard the flux equation as a
system of ODEs over time which can be solved through the multi-step Runge-Kutta
method. This is a family of methods with different predictor mid steps. There is
also an implicit version called IMEX Runge-Kutta which is for stiff equations.

Numerical schemes are all limited by the CFL condition which is a first and
necessary condition for stability. From a physical point of view the CFL
condition ensures that propagation speed of any physical perturbation is always
smaller than numerical speed defined as $\lambda_N = \Delta x/\Delta t$.

\subsection{Discontinuous Galerkin Methods}

DG methods can be considered as numerical methods for the weak formulation of
the equations. Once implemented DG methods reach arbitrary order of accuracy in
space by simply acting on the degree of the expansion polynomials. They allow
for dynamic refinement of mesh and change of polynomial degree. They incorporate
Riemann solver, and they can be easily extended to general unstructured meshes.
The DG method converges exponentially with the order of polynomial used.

The cons of DG method involves a change of philosophy. They are constrained by a
more severe CFL condition than finite volume, and one needs to deal with
oscillations at discontinuities.

Assume a system of PDEs are written in the conservative form. We expand the
solution in polynomials as \eqref{eq:1}. We multiply the equation with a basis
polynomial $\psi_k$ and integrate
\begin{equation}
  \label{eq:9}
  \int_{l_i}\psi_k\frac{\partial U_h}{\partial t}\,dx + \int_{\partial l_i} \psi_kF(U_h)\cdot n\,dS - \int_{l_i}\nabla\psi_k\cdot F(U_h)\,dx = 0
\end{equation}


The CFL condition for DG schemes is, according to (Krivodonova and R.Qin 2013)
\begin{equation}
  \label{eq:5}
  \Delta t < \frac{1}{(2M + 1)}\frac{h}{\left| \lambda_\mathrm{max} \right|}
\end{equation}
while according to (Gottlieb and Tadmor 1991, Radice and Rezzolla 2011) it
should be
\begin{equation}
  \label{eq:6}
  \Delta t < \frac{1}{(M + 1)^2}h
\end{equation}

There are two ways to go around this condition. One is to simply use a very
coarse grid $h$. The second way is local time-stepping where each cell is
updated in time at its own maximum stable time-step.

A modal (hierarchical) basis is a set of basis polynomials of maximum degree $M$
with degree from 0 to $M$. The Legendre polynomials is a example of such a set.
Another possibility is nodal basis. It is a basis of $M + 1$ polynomials of
order $M$. We consider a reference coordinate $\xi\in [0, 1]$ defined between
the grid points. The nodes are zeros of the Gauss-Legendre quadrature. The
polynomials $\psi_l(\xi)$ are Lagrange interpolation polynomials. Having
selected these nodal points we can compute integrals through Gaussian quadrature
\begin{equation}
  \label{eq:7}
  \int_0^1g(\xi)d\xi = \sum_0^M\omega_k\xi_k
\end{equation}

The discontinuous Galerkin scheme is $L_2$ stable, i.e.\
\begin{equation}
  \label{eq:8}
  \int_{\Omega} \partial_t \left( \frac{U^2}{2} \right)\,dx \leq 0
\end{equation}
This property makes DG scheme very robust for e.g.\ high vorticity problems.
However it does not prevent oscillations at discontinuities.

\subsection{Runge-Kutta DG}

We substitute \eqref{eq:1} into the discrete Galerkin equation and get a system
of ODEs in time for the degrees of freedom. This system can be solved through a
standard Runge-Kutta discretization in time.

To first order the RKDG scheme coincides with a first-order finite-volume
scheme. The values of the fluxes at the cell borders can be obtained by solving
a Riemann problem. In principle the solution of such Riemann problems does not
require any spatial reconstruction at the interface between cells. The
efficiency of RK time discretization decreases if the order of accuracy is
higher than 4: the number of intermediate stages becomes larger than the formal
order of accuracy, the so called ``Butcher barrier''.

\section{Lecture 2}

\subsection{ADER DG schemes}

\subsection{Numericl Tests}

\end{document}
