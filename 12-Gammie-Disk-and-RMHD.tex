\documentclass[letterpaper, 11pt]{article}

\usepackage[pdftex]{graphicx}
\usepackage{epstopdf}
\DeclareGraphicsRule{*}{mps}{*}{} 

\usepackage{amsmath, amsthm, amssymb}
\usepackage{listings}
\usepackage{float}
\usepackage{enumerate}
% \usepackage{mystyle}
\usepackage{hyperref}
\usepackage{tikz}
\usepackage{fancyheadings}
\usepackage{tensor}
\usepackage{mathrsfs}
\usetikzlibrary{positioning}
\usetikzlibrary{decorations.pathmorphing}
\usetikzlibrary{arrows}
\usetikzlibrary{decorations.markings}
%\usepackage{fullpage}
\usepackage[left=0.75in, top=1.25in, right=0.75in, bottom=1.25in]{geometry}
\newcommand{\lambdabar}{{\mkern0.75mu\mathchar '26\mkern -9.75mu\lambda}}
\newcommand{\Jmath}{J}

\numberwithin{equation}{section}
\numberwithin{figure}{section}

\begin{document}

\title{Relativistic Astrophysics}
\author{Charles Gammie}
\date{July 25, 2016}

\maketitle

\section{Lecture 1 --- Accretion Disks}

\subsection{Child's Garden of Astrophysical Disks}

Here is a (incomplete) list of astrophysical disks
\begin{itemize}
\item Galactic disk: spiral/elliptical
\item Supermassive BH: quasar/Seyfert/LINER/LLAGN/TDE
\item Stellar mass BH: microquasar/GRB
\item Neutron star: LMXB/HMXB/GRB
\item White dwarf: dwarf nova/nova
\item Protostar: protoplanatory/debris
\item Planet: protolunar disk/planetary rings
\end{itemize}

Consider the galaxy NGC 4258. If we zoom in to $~0.1\,\mathrm{pc}$, we see maser
spots around the central engine. The velocities of these spots fall nicely on a
Keplerian orbit. The combination of the acceleration, velocity, and angular
position of these maser spots allows us to measure to mass of the central
object.

Here let us introduce the first dimensionless parameter of a thin disk
\begin{equation}
  \label{eq:1}
  \frac{H}{R}\ll 1, \quad H = \frac{c_s}{\Omega} \propto T^{1/2}
\end{equation}
For this system this parameter is $\sim 10^{-3}$. This is an example of a thin
disk. The disk of NGC 4258 is not a flat disk, since one can see warps following
the maser spots. The disk is actually made of rings at the positions of the
masers.

This particular disk is relatively far away from the object, and the disk is
heated due to the radiation from the central object.

The next example is the center of our galaxy. We know that the accretion flow
near $\sim 10 r_{g}$ emits synchrotron radiation which is the target of the
Event Horizon Telescope (EHT). The mean free path of Coulomb scattering in Sgr
A* is very large, so the plasma is almost collisionless. This allows us to
introduce the second number
\begin{equation}
  \label{eq:2}
\kappa_n = \frac{\lambda_{mfp}}{R}
\end{equation}
which is about $10^5$ for this system.

The next example is the HL Tau. This is a cold disk of protoplanetary disk. At
$50\,\mathrm{au}$ the temperature is less than $100\,K$. Here we introduce the
third number which is the magnetic Reynolds number
\begin{equation}
  \label{eq:3}
  \mathrm{Re}_M = \frac{c_s H}{\eta}
\end{equation}
In regions of this system that we can measure with ALMA, this number is less
than 1.

The last number we introduce is
\begin{equation}
  \label{eq:4}
  Q = \frac{c_s\Omega}{\pi G \Sigma}
\end{equation}
where $\Sigma$ is the surface density of the disk. When this number is less than
one, self gravity becomes important.

The last example is the moon. When people look at the moon most people do not
see a disk. The theory of the formation of the moon involves an impact on earth,
throwing some mass from the earth which forms a disk around it. This is a system
where the magnetic Reynolds number might play a role.

One more system we want to mention is the SS Cyg which is a Cataclysmic
variable, formed by a massive star overflowing its Roche lobe, feeding a disk
around its companion which is a WD. We have observed this system since 1896 and
it alternates between a quiescent state and an active state. This system may
have a low magnetic Reynolds number at quiescent state, but has a high magnetic
turbulence during its outbursts.

\subsection{Disk Evolution}

Disks are a class of special objects in astrophysics. The angular momentum is
conserved, but its kinetic energy is easily converted to thermal energy due to
various instabilities in the disk, and eventually radiated away. Therefore
understanding the disk evolution is all about understanding the evolution of
angular momentum.

In a thin disk, dynamical equilibrium is reached in a dynamic timescale $\Delta
t \sim \Omega^{-1}$, where
\begin{equation}
  \label{eq:5}
  \Omega = \left( \frac{GM}{R^3} \right)1/2 + O \left( \frac{H}{R} \right)^2
\end{equation}
However it is possible for there to be long living excitations which live in
circular orbits, in addition to tilts and warps in the disk.

Thermal equilibrium is reached when $Q^+ \sim Q^{-}$, when heating rate balances
the cooling rate. The time scale is
\begin{equation}
  \label{eq:6}
  \Delta t \sim \Sigma c_s^2 / Q^+ \sim (\alpha\Omega)^{-1}
\end{equation}
where the $\alpha$ parameter is very important and it describes the intensity of
turbulence in the disk, and the above equation can be taken as the definition.
It relates the thermal time scale to the dynamical time scale. Its value,
usually at $10^{-2}$ means that the thermal time scale is much longer than the
dynamical time scale.

The last time scale is the inflow equilibrium $\dot{M} \sim \mathrm{const}$. The
time scale is
\begin{equation}
  \label{eq:7}
  \Delta t \sim \frac{M_\mathrm{disk}}{\dot{M}} \sim (\alpha\Omega)^{-1}\left( \frac{R}{H} \right)^2
\end{equation}
which is again longer by a factor of the scale height squared.

Lets write down the disk evolution equation
\begin{equation}
  \label{eq:8}
  \partial_t\Sigma = \frac{2}{r}\partial_r \left( \frac{\Omega}{r\kappa^2}\partial_r(r^2W_{r\phi}) - \frac{\Omega}{\kappa^2}\tau \right) + \dot{\Sigma}_\mathrm{ext}
\end{equation}
where $\Sigma$ is the surface density, $\Omega$ the orbital frequency, $\kappa$
the epicyclic frequency which is similar to $\Omega$, $W_{r\phi}$ is the shear
stress, $\tau$ is connected to external torques per area, and finally
$\dot{\Sigma}_\mathrm{ext}$ is the mass gain/loss from infall or from wind loss.
The full derivation of this equation is left as an exercise. The hint is to
start from the angular momentum conservation equation together with conservation
of mass. Another hint is that
\begin{equation}
  \label{eq:9}
  \frac{d}{dr}j = \frac{r\kappa^2}{2\Omega}
\end{equation}
where $j$ is the specific angular momentum.

The standard model for disks is the $\alpha$ disk model. It was introduced by
(Shakura \& Sunyaev 1973) and (Linden-Bell \& Pringle 1974). It adopts simple
scaling argument for diffusion of angular momentum by turbulence, by just
modeling the turbulent viscosity
\begin{equation}
  \label{eq:10}
  \nu \sim \alpha c_s H
\end{equation}
In classical $\alpha$ disk models one ignores external torques, infall/winds,
etc. Thus one throws away the last two terms in the disk equation, with only the
term with $W_{r\phi}$ surviving.

In the $\alpha$ disk model we assume the disk is thin and Keplerian $\Omega
\approx (GM/r^3)^{1/2}$. We also assume vertical equilibrium $H \sim
c_s/\Omega$. The opacity is assumed to be related to temperature and density $\kappa \sim
\kappa_0\rho^aT^{b}$. We have a prescription of vertical integration of the
density $\Sigma \sim 2\rho H$, which is a crude integral. We also estimate the
optical depth to be $\tau \sim \Sigma \kappa / 2$. The surface temperature is
determined by the fact that heating by viscous processes $F = \sigma
T_\mathrm{eff}^4 \sim (9/8)\Sigma \nu \Omega^2$, where $T_\mathrm{eff}^4 \sim
\sigma T^4/\tau$. Finally in the steady state $\dot{M} = 3\pi \Sigma \nu$.

As an example, consider a steady state disk around a stellar mass BH. In the
inner zone, radiation pressure is much larger than the gas pressure, and
electron scattering dominates the opacity. We can find that
\begin{align}
  T &\sim 4.3\times 10^7 \alpha^{-1/4}m^{-1/4}x^{-3/8}\,K \\
  \Sigma &\sim 0.4 x^{3/2}\alpha^{-1}\dot{m}^{-1}\,\mathrm{g/cm^2} \\
  H/r &\sim 10 \dot{m}x^{-1}
\end{align}

Now lets come back to the disk evolution equation and ask what are $W_{r\phi}$,
$\tau$, and $\Sigma_\mathrm{ext}$. Part of the answer only lies in the global
analysis of turbulence in the disks.

\subsection{Turbulence in Disks}

In $\alpha$ disk model the turbulent diffusion of angular momentum is simply
postulated. Possible sources of turbulence include magnetorotational instability
(MRI), gravitational instability, zombie vortex instability, subcritical
baroclinic instability, and vertical shear instability. The last three are
related to instabilities in fluids. The zombie vortex instability is generated
due to vortices generated in the disk sharing angular momentum through sound
waves. The subcritical baroclinic instability is loosely related to the
baroclinic instability that drives weather. The vertical shear instability is
when shearing motion is faster than orbital motion and driving the flow of
angular momentum at sound speed.

We will focus on the MRI instability (Balbus \& Hawley 1991). This is due to
local, linear instability of weakly magnetized disks driven by exchange of
angular momentum. The condition of the instability does not involve boundary
conditions and occur in a localized region of the disk, and can be treated using
the WKB approximation.

We will start with a simple mechanical analogy where 2 masses connected by a
spring are in orbit around a central body. The orbital frequency is given by
Keplerian motion $\Omega^2 = GM/R^{3}$. The masses can be thought as fluid
elements, and spring can be a model of magnetic field line tension which
provides restoring force. We denote the frequency of the spring as $\gamma$.

Lets setup $x$-$y$ coordinates for the orbital motion where $x$ is in radial
direction, and $y$ points opposite to the orbital motion. We know that
\begin{equation}
  \label{eq:11}
  \ddot{x} = -2\Omega \dot{y} + 3\Omega^{2}x - \gamma^2x
\end{equation}
The first term is the Coriolis force, and the second combines the centrifugal
force and the gravitational force in an expansion, and the third term is the
spring force. The second equation is
\begin{equation}
  \label{eq:12}
  \ddot{y} = 2\Omega \dot{x} - \gamma^{2}y
\end{equation}

As a first go, lets set $\gamma = 0$ and assume $x$ and $y$ scale with
$e^{-i\omega t}$, the equations become
\begin{align}
  \label{eq:13}
  -\omega^2x &= -2\Omega(-i\omega) y + 3\Omega^2 x \\
  -\omega^2y &= 2\Omega(-i\omega)x
\end{align}
This simply gives $\omega^2 = \Omega^2$ which means that the natural modes
oscillate at the orbital frequency.

Now lets add the $\gamma$ term, and we get
\begin{align}
  \label{eq:14}
  -\omega^2x &= -2\Omega(-i\omega) y + 3\Omega^2 x - \gamma^2x \\
  -\omega^2y &= 2\Omega(-i\omega)x - \gamma^2y
\end{align}
And solving the dispersion relation gives
\begin{equation}
  \label{eq:14}
  \omega^4 - \omega^2(\Omega^2 + 2\gamma^2) + \gamma^2(\gamma^2 - 3\Omega^2) = 0
\end{equation}
Notice that there is a zero solution condition when $\gamma^2 = 3\Omega^2$, and
it is a transition from positive to negative frequency square. If we plot
$\omega^2/\Omega^2$ vs $\gamma^2/\Omega^2$, there are two sets of roots. One is
simply spring motion. The other starts with $\omega^{2} = 0$, and between
$\gamma^2 = 0$ and $3\Omega^2$ we have $\omega^2 < 0$ where there is an
instability. The fastest growing mode of the instability is at
$\gamma^2/\Omega^2 = 15/16$ and $\omega^2/\Omega^2 = -9/16$.

This is directly analogous to an MHD problem. Lets zoom in on the disk and
assume it is penetrated by a weak vertical magnetic field. This instability is
related to the magnetic instability where the field is bent, and the fluid
element above and below are coupled by Alfven waves. In this case we have
\begin{equation}
  \label{eq:15}
  \gamma^2 \to (\mathbf{k}\cdot \mathbf{v}_A)^2
\end{equation}

Lets list some facts about MRI linear theory. The ideal fluid instability only
requires $d\Omega^2/dr < 0$. The maximum growth rate is $(3/4)\Omega$ for
Keplerian motion. The fastest growing mode is given above. We know that there is
a local instability when there is vertical field, but a local instability also
occurs even when there is only toroidal field.

However linear theory doesn't tell us what happens in the regime when linear
growth saturates. To study this regime we need simulations. In simulations
people have used local or global setups. There are models with explicit small
dissipation terms, or ILES which self-consistently evolves turbulence cascade
from the scale where energy is injected down to the scale of dissipation. There
are also isothermal models with equation of state $P = c_s^2\rho$, or one can
use energetically self-consistent equation of state.

Some simulations were shown\dots

Simulations of MRI told us a few things. In 2D MRI leads to turbulence and
$\alpha$. We know that in 2D MRI does not converge, so $\alpha$ depends on
resolution. In 3D MRI also leads to turbulence and $\alpha$, but sometimes it
does converge. We know that it converges when explicit dissipation is used.
There are issues with ILES models with unstratified shearing boxes. We also
learned that $\alpha$ depends on many parameters. It depends on the local height
$z$, and it increases away from the central plane. It also depends on the
magnitude of the vertical magnetic field that threads the disk $\left<
  B_z\right>$. When the vertical magnetic field increases the viscosity also
increases. It depends on the magnetic Reynolds number and $\mathrm{Pr}_{M} =
\nu/\eta$.

\subsection{Current Problems in Disk Theory}

Ran out of time\dots

\section{Lecture 2 --- Relativistic MHD}

\subsection{When is Relativistic MHD required?}

The first situation when we need relativistic MHD is when one of the wave speeds
is of order $c$. The characteristic wave speeds are usual one of $\mathbf{v}$,
$\mathbf{v}_A$, and $\mathbf{v}_{s}$. The nonrelativistic sound speed can be
defined as
\begin{equation}
  \label{eq:16}
  c_{s,NP}^2 = \gamma\frac{P}{\rho}
\end{equation}
which blows up at high $\gamma$ and low $\rho$. If we do it correctly with
relativistic theory we can find that
\begin{equation}
  \label{eq:17}
  c_{s,R}^2 = \frac{\gamma P}{\rho + \frac{\gamma}{\gamma - 1}\frac{P}{c^2}} = \frac{c_{s,NR}^2}{1 + c_{s,NR}^2/(\gamma - 1)/c^2}
\end{equation}
In relativistic limit we have $\gamma = 4/3$, $\rho = 3P$, and we have sound
speed as $c_s^2 = c^2/3$. If we still use ideal gas law $P = nkT$, and $n = \rho
/ m$ then since we have in the relativistic limit $P \sim \rho c^2$
\begin{equation}
  \label{eq:18}
  \theta = \frac{kT}{mc^2} \sim 1
\end{equation}
This dimensionless number for electron $\theta_{e} = 1$ means $T \sim 5.9\times
10^7\,K$, and $\theta_p = 1$ means $T \sim 1.1 \times 10^{13}\,K$.

Lets look at Alfven speed. For nonrelativistic case it is defined as $v_A =
B/\sqrt{4\pi\rho}$, which blows up when $\rho \to 0$. We are saved in
relativistic case by
\begin{equation}
  \label{eq:19}
  v_{A, R} = \frac{\left| \mathbf{B} \right|}{\sqrt{4\pi\rho + B^2/c^2}}  = \frac{v_{A,NR}}{\sqrt{1 + v_{A,NR}^2/c^2}}
\end{equation}
Therefore in relativistic case $B$ field inertial comes into play. The
dimensionless ratio here is
\begin{equation}
  \label{eq:20}
  \sigma = \frac{B^2}{4\pi\rho c^2}
\end{equation}

Without looking at speeds, we also need relativistic hydro when
\begin{equation}
  \label{eq:21}
  \phi \sim \frac{GM}{rc^2} \sim 1
\end{equation}
which accounts for gravity effects.

\subsection{Basic equations}

Lets write down the equations for relativistic MHD in conservative form in terms
of fluxes and conserved quantities
\begin{equation}
  \label{eq:22}
  \partial_t\mathsf{U} = -\nabla\cdot \mathsf{F} + \mathsf{S}
\end{equation}
Now we need to be careful about what are the conserved quantities and what is
the divergence operator. What we are going to do is write down the
nonrelativistic versions and use them to motivate the transition to relativistic
version.

The mass conservation equation looks like
\begin{equation}
  \label{eq:23}
  \partial_t\rho = -\nabla\cdot(\rho \mathbf{v}) = -\partial_i(\rho v_i)
\end{equation}
The relativistic generalization is straightforward:
\begin{equation}
  \label{eq:24}
  \nabla_{\mu}(\rho u^{\mu}) = 0
\end{equation}
where $u^{\mu} = dx^{\mu}/d\tau$. We need to define density a bit more
carefully. $\rho$ is defined using the particle density in a frame that is
comoving with the plasma. The $\nabla_{\mu}$ operator is the covariant
derivative operator which was covered in lecture by Lehner. We define the gamma
factor of the particles as $u^0 = \Gamma$, and $u^i = \Gamma v^{i}$. The 4
velocity is normalized with $u^{\mu}u_{\mu} = -1$.

The mass conservation equation can be expanded out in a coordinate in the
following form:
\begin{equation}
  \label{eq:25}
  \partial_t(\sqrt{-g}\rho u^t) = -\partial_i(\sqrt{-g}\rho u^i)
\end{equation}
where $g$ is the determinant of the metric tensor. This is because the
4-divergence can be written as $1/\sqrt{-g}\partial_{\mu}(\sqrt{-g}\rho
u^{\mu})$. This equation now looks like the form $\partial_t\mathsf{U} =
-\nabla\cdot \mathsf{F}$.

Now we look at energy momentum conservation equations. The nonrelativistic form
looks like
\begin{align}
  \label{eq:26}
  \partial_t \left( \frac{1}{2}\rho v^2 + U + \frac{B^2}{8\pi} \right) &= -\partial_i \left( \frac{1}{2}\rho v^2v_i + (U + P)v_i + \frac{1}{4\pi} \left( B^2v_i - (\mathbf{B}\cdot \mathbf{v})B_i \right) \right) \\
  \partial_t(\rho v_i) &= -\partial_j\Pi_{ij}
\end{align}
where
\begin{equation}
  \label{eq:27}
  \Pi_{ij} = \rho v_iv_j + P\delta_{ij} + \frac{B^2}{8\pi}\delta_{ij} - \frac{B_iB_j}{4\pi}
\end{equation}
In relativistic case these two equations combine to have the following simple
form
\begin{equation}
  \label{eq:28}
  \nabla_{\mu} T^{\mu\nu} = 0
\end{equation}
All the physics is now in the stress-energy tensor. This covariant derivative
can be expanded in the coordinate form
\begin{equation}
  \label{eq:29}
  \partial_t(\sqrt{-g}\tensor{T}{^t_{\mu}}) = -\partial_i \left( \sqrt{-g}\tensor{T}{^i_{\mu}} \right) + \sqrt{-g}\tensor{T}{^{\kappa}_{\lambda}}\tensor{\Gamma}{^{\lambda}_{\mu\kappa}}
\end{equation}
where $\Gamma$ is the connection related to $\partial g_{\mu\nu}$.

Now lets write down the stress-energy tensor
\begin{equation}
  \label{eq:30}
  T_{\mu\nu} = (\rho + U + P + b^2)u_{\mu}u_{\nu} + \left(P + \frac{b^2}{2} \right)g_{\mu\nu} - b_{\mu}b_{\nu}
\end{equation}
where $b_{\mu}$ has spatial components $b_i$ which is just the spatial magnetic
field. In the fluid rest frame this tensor is diagonal, and if we choose $B$ to
be aligned with $z$ direction then we have
\begin{equation}
  \label{eq:31}
  T_{\mu\nu} = \text{diag} \left( \rho + U + \frac{b^2}{2}, P + \frac{b^2}{2}, P + \frac{b^2}{2}, P - \frac{b^2}{2} \right)
\end{equation}
Notice that the first term in the stress energy tensor has energy terms, so we
already take into account the temperature of the plasma in the energy.

Lets look at the magnetic field equation $\partial_t\mathbf{B} =
\nabla\times(\mathbf{v}\times \mathbf{B}) = -\partial_j(v_{j}B_i - v_iB_j)$. It
is easy to generalize this by promoting $\partial_i$ to $\nabla_i$ and $v$, $B$
to $u$ and $b$, we have $\nabla_{\mu}(u^{\mu}b^{\nu} - u^{\nu}b^{\mu}) = 0$, and
in coordinate basis it looks like:
\begin{equation}
  \label{eq:32}
  \partial_t \left( \sqrt{-g}(u^tb^i - u^ib^t) \right) = -\partial_i \left( \sqrt{-g} (u^jb^i - u^i b^j)\right)
\end{equation}
This equation has 3 components, and the 4th component is simply the no-monopole
constraint
\begin{equation}
  \label{eq:33}
  \partial_i(\sqrt{-g}\tilde{B}^i) = 0
\end{equation}
where we define $\tilde{B}^i = b^iu^t - u^ib^t$.

Now we can combine all these equations and write down what is $\mathsf{U}$,
$\mathsf{F}$, and $\mathsf{S}$. The perimeter variables we want to use are
$\rho$, $U$, $\tilde{B}^i$, and $u^i$. However $u^i$ are not convenient in e.g.\
black hole ergospheres, so we define
\begin{equation}
  \label{eq:34}
  \tilde{u}^i = (\tensor{g}{^i_{\mu}} + n^in_{\mu})u^{\mu}
\end{equation}
We can transform these into conserved variables $\mathsf{U}$ as
\begin{equation}
  \label{eq:35}
  \left( \sqrt{-g}\rho u^t, \sqrt{-g}(\rho + U + P + b^2)u^tu_t + (P + b^2/2)g^t_t - b^tb_t, \sqrt{-g}\tilde{B}^{i} \right)
\end{equation}

Lets look at the Kerr metric. The Boyl-Lindquist coordinates is $t$, $r$,
$\theta$, $\phi$, and we have $\partial_{\phi} = 0$ and $\partial_t = 0$. These
two represent two Killing vectors $\xi$ which satisfy
\begin{equation}
  \label{eq:36}
  \nabla_{\mu}\xi_{\nu} + \nabla_{\nu}\xi_{\mu} = 0
\end{equation}
For example $\xi_t^{\mu} = (1, 0, 0, 0)$, and similar for $\xi_{\phi}^{\mu}$. If
we define conserved current as $J^{\mu} = T^{\mu}_{\nu}\xi^{\nu}$, then we
automatically have $\nabla_{\mu}J^{\mu} = 0$ due to the definition of Killing
vectors.

\subsection{Numerical techniques}

There are about a dozen codes out there that can integrate relativistic MHD
equations. We start from perimeter variables $\mathsf{P}^n$ at timestep $n$ and
find the conserved variables $\mathsf{U}^n$, integrate the equations to get
$\mathsf{U}^{n+1}$ and do an inverse transform to get $\mathsf{P}^{n+1}$ which
usually involves numerically solving systems of nonlinear equations.

In HARM we don't solve Riemann problems, and we need to estimate $v_{\pm}$. It
is diffusive in comparison to Athena++, but it is much simpler. We still need to
keep $\nabla\cdot \mathbf{B} = 0$. One could simply do nothing and hope for the
best, but that proved to not work. Another solution is to use the so-called
constraint transport, which embed this constraint into the transport equation.
Yet another approach is divergence cleaning, which solves periodically the
closest $B$ field which has no divergence. A related approach is to evolve an
additional field which causes in the induction equation an extra term that
relaxes deviation from the constraint to zero over time. The final one is simply
to evolve the vector potential, which is guaranteed to have $B$ field divergence
free.

The procedures to compute the time evolution of the conserved variables are now
pretty well understood. However there is no analytic approach to get the
fluxes so that is a challenge. It is also understood now that the inversion
process in the end to get the perimeter variables does not dominate the
numerical complexity.

One problem that is common for both RMHD and MHD is that sometimes evolution
takes conserved quantities to nonphysical values. Some codes simply inspect the
quantities and for example put a floor on density values to prevent it from
going negative. The reason one could do this is that this kind of scenario shows
up in extreme cases and the change of a small negative density to zero does not
change the system too much. In HARM it is imposed that
\begin{equation}
  \label{eq:37}
  \rho > \rho_\mathrm{min},\quad U > U_\mathrm{min},\quad -n^{\mu}u_{\mu} < \Gamma_\mathrm{max}
\end{equation}
In the procedure of this fixup, one introduces error to the energy and
temperature of the system, which is somewhat inevitable.

Feynman said: ``The first principle is that you must not fool yourself, and you
are the easiest person to fool''. One need to test the code to show correct
convergence behavior. Typical code tests include linear wave convergence which
only tests the code in the linear regime. Another test is Orszag-Tang vortex
(Gammie et.\ al.\ 2003). We tested the code against another code VAC. One could
also test against changing resolution. We also tested it with Kerr inflow
(reversed Parker wind) which has an ``exact'' solution for us to compare with.

Field loop advection problem and Komissarov's sadistic explosion problem\dots

\subsection{Beyond ideal MHD}

In low $\dot{M}$ BH accretion flows we have $\lambda_{mfp,\parallel}$ for
Coulomb scattering much larger than the gravitation scale $GM/c^2$. However the
perpendicular mean free path is still very small. The associated viscosity $\nu
\sim v_\mathrm{th}\partial_{mfp}$ would be highly anisotropic. Particles can
carry momentum along the field lines but not across field lines. There is also
anisotropic thermal conduction. Since Coulomb collision is not very efficient
the electron and ion temperatures will decouple.

One theory include the viscosity and thermal conduction is the covariant
extended MHD (Chandra+ 2015). This amounts to adding additional terms to the
stress-energy tensor. In this model we add heat flux $q^{\mu} = q b^{\mu}$,
where $b^{\mu}$ is a unit spacelike four-vector parallel to the $b$ field. There
is also possible momentum transport parallel to the field
\begin{equation}
  \label{eq:38}
  \tau^{\mu\nu} = -\Delta P \left( b^{\mu}b^{\nu} - (1/3)h^{\mu\nu} \right)
\end{equation}
where $h^{\mu\nu}$ is the projection tensor perpendicular to $u^{\mu}$ and
$u^{\mu}$ is the 4-velocity. This is similar to the Braginskii stress tensor.

A naive theory setting $q \sim \nabla T + Ta$ which is a combination of
temperature gradient and 4-acceleration is unstable, mostly due to the
4-acceleration term. One needs to promote $q$, $\Delta P$ to dependent variables
and evolve them. One need to have equations for $q_0$, $\Delta P_0$, and
relaxation equations for these. Finally one needs to have closure relation for
$\chi$ and $\nu$.

An inspirational paper is (Komissarov 1999) called ``A Godunov-type scheme for
relativistic magnetohydrodynamics''.

\end{document}
