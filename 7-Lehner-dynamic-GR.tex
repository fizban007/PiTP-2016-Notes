\documentclass[letterpaper, 11pt]{article}

\usepackage[pdftex]{graphicx}
\usepackage{epstopdf}
\DeclareGraphicsRule{*}{mps}{*}{} 

\usepackage{amsmath, amsthm, amssymb}
\usepackage{listings}
\usepackage{float}
\usepackage{enumerate}
% \usepackage{mystyle}
\usepackage{hyperref}
\usepackage{tikz}
\usepackage{fancyheadings}
\usepackage{tensor}
\usepackage{mathrsfs}
\usetikzlibrary{positioning}
\usetikzlibrary{decorations.pathmorphing}
\usetikzlibrary{arrows}
\usetikzlibrary{decorations.markings}
%\usepackage{fullpage}
\usepackage[left=0.75in, top=1.25in, right=0.75in, bottom=1.25in]{geometry}
\newcommand{\lambdabar}{{\mkern0.75mu\mathchar '26\mkern -9.75mu\lambda}}

\numberwithin{equation}{section}
\numberwithin{figure}{section}

\begin{document}

\title{GR meets astrophysics}
\author{Luis Lehner}
\date{July 20, 2016}

\maketitle

\section{Lecture 1}

We want to make connection between GR and the plasma astrophysics. Why do we
need this when we have gravitational waves now? There are a lot of limitations
of gravitational waves. It is limited due to characteristics of signals and
detectors. At higher frequency noise grows linearly, and sky location is very
poor. Strong gravity effects in astrophysics are all tied to some
analytical models, and are limited in scope. We also are interested in the EM
connections to the gravitational wave signals.

Therefore we want to study mergers of compact objects within surrounding plasma.
Two binary non-rotating black holes disturbs the spacetime enough that it
launches jets from the surrounding plasma and allow the plasma to tap into the
kinetic energy of the rotation.

We will be talking about the challenges and problems simulating and controlling
dynamic space time. Let's go slowly from the beautiful theory of Einstein and
put it into the computer to compute it dynamically.

\subsection{Well-posedness}

``Theorems are permanent, tricks are ephemeral.''

We start with the theorem by Hadamard: any problem about a physical system
should be well posed. By well posed (WP), we mean:
\begin{itemize}
\item Solution exists
    \item The solution to the problem is unique
    \item The solution must depend continuously on initial and boundary data
\end{itemize}
By the last statement we mean that the magnitude of solution in a general sense
should
\begin{equation}
  \label{eq:1}
  \left| u \right|_T \leq \left| u \right|_{t = 0} K e^{\beta T}
\end{equation}
where $K$ and $\beta$ should not depend on initial or boundary data. If any
problem we are solving does not satisfy any of the statements we should throw it
away immediately.

Lets talk about the sufficient and necessary conditions for WPness. If we can
pose the problem in this form
\begin{equation}
  \label{eq:2}
  u_{,t} = \sum A^i\partial_iu + \text{``Rest''}
\end{equation}
then the sufficient condition is that the matrix $A$ is diagonalizable, and the
eigenvalues are real.

A simple example is $u_{,tt} = u_{,xx}$. If we define $f = u_{,x}$ and $g =
u_{,t}$ then we can define the dynamic variables as $u, f, g$ and it satisfy the
sufficient condition.

Consider now this system
\begin{equation}
  \label{eq:3}
  u_{,t} = u_{,x} + v_{,x},\quad v_{,t} = u_{,x}
\end{equation}
This system has $A$ in a Jordan block form. It turns out that this system admits
the solution in the form of
\begin{equation}
  \label{eq:4}
  u = DF_1(t + x)\cdot t + F_2(t + x),\quad v = F_1(t + x)
\end{equation}
the first part of $u$ grows linearly with time. This solution is okay and
satisfies our WPness condition, but the growth with time is alarming. This is
called a weakly hyperbolic system, and we now need to care about the ``Rest'' of
the problem. For example, if we couple this, and add $u$ to $v_{,t}$, then the solution of
$v$ becomes
\begin{equation}
  \label{eq:5}
  v = c_1e^{c_1(t+x)} + c_2e^{t\sqrt{c_2}}
\end{equation}
and this solution is exponentially unstable to changes to initial condition,
which is a badly posed problem. We especially want to avoid this in a numerical
system because the initial conditions are never exact.

\subsection{Crash Course in GR}

How do we make sure our complex problem is well posed when such a simple one can
be bad? Lets consider the Einstein equation
\begin{equation}
  \label{eq:6}
  G_{ab} = 8\pi T_{ab}
\end{equation}
The left hand side part has form of $\partial^{2}g$, whereas the right hand side
depends on the fluid quantities $\mathcal{F}$ and $g$. We also have the
constraint equation of the fluid $\nabla_aT^{ab} = 0$, which concerns $\partial
\mathcal{F}$ and $\partial g$. For our purposes they are decoupled, so we are
allowed to treat $T_{ab}$ as the ``Rest''.

Lets say some words about gravity quickly. We have the metric tensor which
describes the spacetime manifold $g_{ab}$. This determines a length
\begin{equation}
  \label{eq:7}
  ds^2 = g_{ab}dx^adx^{b}
\end{equation}
Note that we use Einstein summation notation to add the indices $a$ and $b$.
Both indices take values $0, 1, 2, 3$ where $0$ is for time part and others are
for spatial part.

If we have a curved manifold, when we take derivatives we need to take into
account the change of manifold in addition to the change of the field. We define
the covariant derivative $\nabla_{a}$ which does exactly that. We have by definition
\begin{equation}
  \label{eq:8}
  \nabla_ag_{bc} = 0
\end{equation}
The covariant derivative of a vector is defined as
\begin{equation}
  \label{eq:9}
  \nabla_av^b = \partial_av^{b} + \tensor{\Gamma}{^{b}_{ac}}v^{c}
\end{equation}
Some other definitions are
\begin{equation}
  \label{eq:10}
  \nabla_a f = \partial_{a}f,\quad \nabla_av_{b} = \partial_av_{b} - \tensor{\Gamma}{^{c}_{ab}}v_{c}
\end{equation}

Now we need to define what the $\Gamma$ symbol is. It is called the Chrsitoffel
symbol
\begin{equation}
  \label{eq:11}
  \tensor{\Gamma}{^{a}_{bc}} = \frac{1}{2}g^{ae}(g_{be,c} + g_{ce,b} - g_{bc,e})
\end{equation}
Now we can define the so-called Riemann curvature tensor:
\begin{equation}
  \label{eq:12}
  \tensor{R}{^a_{bcd}} = \partial_{c}\tensor{\Gamma}{^a_{bd}} - \partial_{d}\tensor{\Gamma}{^a_{bc}} + \tensor{\Gamma}{^{e}_{bd}}\tensor{\Gamma}{^{a}_{ec}} - \tensor{\Gamma}{^{e}_{bc}}\tensor{\Gamma}{^{a}_{ed}}
\end{equation}
This curvature tensor describes the curvedness of the spacetime manifold.

If we have two world lines along the manifold, and the acceleration between them
can be characterized by the Riemann tensor
\begin{equation}
  \label{eq:13}
  a^a = -\tensor{R}{^{a}_{cbd}}\xi^{c}\xi^dL^{b}
\end{equation}
where $\xi^{c} = dx^{c}/d\lambda$ $L^{b}$ is the distance between the world
line. A funny fact about the curved manifold is that derivatives depend on
paths, and the order of taking them, and the Riemann tensor characterizes this:
\begin{equation}
  \label{eq:14}
  \left[ \nabla_{b}\nabla_{c} - \nabla_{c}\nabla_{b} \right] v_{a} = \tensor{R}{^{d}_{abc}}v_{d}
\end{equation}
We finally define the Einstein tensor
\begin{equation}
  \label{eq:15}
  G_{ab} = R_{ab} - \frac{1}{2}g_{ab}R,\quad R_{ab} = \tensor{R}{^{d}_{adb}},\quad R = g^{ab}R_{ab}
\end{equation}

\subsection{Specializing into a Formalism}

Now lets get into how to solve these equations in the computer. In a sense our
problem is ill formed from the get go because everything depends on our choice
of coordinates. So we need to find a formulation.

The one we introduce now is called the ADM formulation. Given a spacetime we can
foliate it with space-like slices $\Sigma_{t_i}$. A way to understand space-like
hypersurfaces is that its norm at any place $n^{a}$ is time-like: $n_an^a = -1$.
Here we implicitly choose our signature to be $(-1, 1, 1, 1)$ so that the
Minkowski spacetime is $ds^2 = -dt^2 + dx^2 + dy^2 + dz^2$. Time-like vectors
are inside the lightcone and space-like vectors are outside.

We can now label these hypersurfaces with $t$ equal to constant. Because the
hypersurfaces have constant $t$, we can define $n_{a} = \nabla_at
= \partial_at$. If we go from the hypersurface of $t_{1}$ to the next
hypersurface of $t_2$. For a typical observer, he will not measure really time
$t_2 - t_1$, but some time proportional to that:
\begin{equation}
  \label{eq:16}
  \tau = \alpha(t_2 - t_1)
\end{equation}
where $\alpha$ is called the lapse function. The vector that points from a point
$(t_1, x, y, z)$ to a point $(t_2, x, y, z)$ is what we call $\partial_{t}$.
However this may not coincide with the vector $n_{a}$ we defined above, so we
define a new vector
\begin{equation}
  \label{eq:17}
  \beta^a = \partial_t^a - n^a
\end{equation}
which is called the shift vector.

Now we can split up the metric tensor
\begin{equation}
  \label{eq:18}
  ds^2 = -\alpha^2dt^2 + h_{ij}\left( dx^i + \beta^idt \right)(dx^j + \beta^jdt)
\end{equation}
where $h_{ij}$ will be the metric on the hypersurface.

We can find the relationship between the metric tensor of the individual
hypersurfaces and the full metric:
\begin{equation}
  \label{eq:19}
  h_{ab} = g_{ab} + n_an_{b}
\end{equation}
This metric has several properties:
\begin{itemize}
\item $\tensor{h}{_a^{b}}\tensor{h}{_{b}^{c}} = \tensor{h}{_{a}^{c}}$
    \item $h_{ab}n^a = 0$
    \item $h_{ab}S^{b} = S_{a}$
\end{itemize}
where $S^{b}$ is a spacelike vector. So $h_{ab}$ is like a projection operator
onto the hypersurface.

We need one more concept so that we can foliate the spacetime. This is the
so-called extrinsic curvature
\begin{equation}
  \label{eq:20}
  K_{ab} = -\perp \perp \nabla_{a}n_{b} = -\tensor{h}{_{a}^{\alpha}}\tensor{h}{_{b}^{\beta}}\nabla_{\alpha}n_{\beta}
\end{equation}
This is a useful quantity to measure the rate of change of the hypersurface when
embedded into the whole manifold as one goes with the flow. It can be shown that
\begin{equation}
  \label{eq:21}
  \partial_t h_{ab} = -2\alpha K_{ab} + \beta^l\partial_lh_{ab} + h_{al}\partial_{b}\beta^{l} + h_{bl}\partial_{a}\beta^{l}
\end{equation}

Now we are equiped to look at numerical relativity. One takes the Einstein
equations and take different projections. If we take a vector and dot it
$v^{a}n_{a} = v_{n}$ to get the component along the direction of $n$. We can
also dot it with $h$ to get its projection to the hypersurface. If we dot the
Einstein equation with $n^an^{b}$ then we get
\begin{equation}
  \label{eq:22}
  \tensor[^{(3)}]{R}{} + K^{2} - K_{ij}K^{ij} = 16\pi\rho
\end{equation}
Here $\rho$ is the energy density that is measured by the observers that are
moving normal to the spatial hypersurface, $\rho = T_{ab}n^{a}n^{b}$.
$\tensor[^{(3)}]{R}{}$ is the intrinsic curvature of the hypersurface.

We can also imagine dotting the Einstein equation with
$n_{a}\tensor{h}{_{c}^{b}}$:
\begin{equation}
  \label{eq:23}
  D_{b}(K^{ab} - h^{ab}K) = 8\pi J^{a}
\end{equation}
where $D$ is the intrinsic covariant derivative along the hyper surface
$D_{b}h_{ac} = 0$. Again $J$ is the momentum measured by the same observers.

These two equations have no time derivatives and are called constraint
equations. We have one final projection by $\perp\perp$
\begin{equation}
  \label{eq:24}
  \begin{split}
    \partial_tK_{ab} &= \beta^l\partial_lK_{ab} + K_{al}\partial_{b}\beta^l + K_{bl}\partial_a\beta^{l} \\
    &- D_{a}D_{b}\alpha + \alpha \left[ \tensor[^{(3)}]{R}{_{ab}} + K K_{ab} - 2K_{ad}\tensor{K}{^d_{b}} - 8\pi \left( S_{ab} - \frac{h_{ab}}{2}(S - \rho) \right) \right]
  \end{split}
\end{equation}
We also have
\begin{equation}
  \label{eq:25}
  \partial_th_{ab} = \beta^l\partial_lh_{ab} + h_{al}\partial_{b}\beta^{l} + h_{bl}\partial_{a}\beta^{l} - 2\alpha K_{ab}
\end{equation}
From these we know that the spatial curvatures have second order time
derivatives, which makes it into a hyperbolic system, but the rest unfortunately
turns the system into weakly hyperbolic.

\subsection{Overcoming the Weakly Hyperbolic Problem}

There is a whole bunch of choices of $\alpha$ and $\beta$, but lets first
discuss the problem that this system is weakly hyperbolic. Lets take a step back
and write down $R_{ab}$. Notice that taking the trace of the Einstein equation
gives $R = -T$. So we can write 
\begin{equation}
  \label{eq:27}
  R_{ab} = T_{ab} - \frac{1}{2}g_{ab}T
\end{equation}

The Ricci tensor has a very provocative form
\begin{equation}
  \label{eq:28}
  R_{ab} \longrightarrow g^{cd}g_{ab,cd} - 2\nabla_{(a}\Gamma_{b)} + \Gamma\Gamma + \partial g \partial g + \dots
\end{equation}
Since we are interested in the hyperbolic part, the first term is perfect
which is like the wave equation, but the second term is nasty. Due to human
nature we want to throw away things we don't like, so lets try to choose
$\alpha$ and $\beta$ such that $\Gamma_{b} = 0$ then the second term is zero.
Does this do the job?

Suppose we have the equation
\begin{equation}
  \label{eq:29}
  \Box \phi = \sum A \partial_{J}\phi \partial_R\phi
\end{equation}
which is actually what happens with the above equation for $R_{ab}$ and it turns
out it has nice properties. Another problem people found is that $\Gamma_{b} =
0$ gives $\nabla_a\nabla^{a}x^{b} = 0$ which means coordinates fly away at speed
of light.

Now do we need to give up our coordinates to solve the problem? What we do is
to put a source $\Gamma_{b} = H_{b}$, so that $-\nabla_{a}\nabla^{a}x^{b} =
H^{b}$. However now $H_{b}$ shows up in the equation, but if we have a
hyperbolic system for $H$ where $\nabla_a\nabla^{a}H_{b} = \text{"something"}$
then we can close the system.

The remaining challenge is to satisfy the constraints, since a small departure
from the constraint due to numerics then we might go completely away from the
solution. Therefore we need to find a way to bring the solution back to the
constraint hypersurface when things go wrong.

Imagine at some point we choose the coordinates as discussed above, and there is
a deviation $C^{a} = H^{a} + \Gamma^{a}$, then our system becomes
\begin{equation}
  \label{eq:30}
  R_{ab} - \nabla_{(a}C_{b)} = 0
\end{equation}
we can substitute the constraint equation $\nabla^{b}G_{ab} = 0$ into the above
equation we get
\begin{equation}
  \label{eq:31}
  0 = \nabla^{a}\nabla_{a}C_{b} + C^{a}\nabla_{(b}C_{a)}
\end{equation}
so we have a wave equation for $C$. In theory if we start from zero then it will
remain zero. But in reality we do see problems.

Consider instead
\begin{equation}
  \label{eq:31}
  0 = \nabla^{a}\nabla_{a}C_{b} + C^{a}\nabla_{(b}C_{a)} + \gamma_0 \left[ t_{(a}C_{b)} - \frac{1}{2}g_{ab}t^{c}C_{c} \right]
\end{equation}
where $\gamma^0$ is a numerical parameter and $t$ a timelike vector field. If we
do the same thing as above we have now
\begin{equation}
  \label{eq:32}
  0 = \nabla^{a}\nabla_{a}C_{b} - 2\gamma_{0}\nabla^{a}\left[ t_{(a}C_{b)} \right]
\end{equation}
This is now a damped wave equation and will damp away deviation from the
constraint on a time scale determined by $\gamma_{0}$.




\end{document}
