\documentclass[letterpaper, 11pt]{article}

\usepackage[pdftex]{graphicx}
\usepackage{epstopdf}
\DeclareGraphicsRule{*}{mps}{*}{} 

\usepackage{amsmath, amsthm, amssymb}
\usepackage{listings}
\usepackage{float}
\usepackage{enumerate}
% \usepackage{mystyle}
\usepackage{hyperref}
\usepackage{tikz}
\usepackage{fancyheadings}
\usepackage{tensor}
\usepackage{mathrsfs}
\usetikzlibrary{positioning}
\usetikzlibrary{decorations.pathmorphing}
\usetikzlibrary{arrows}
\usetikzlibrary{decorations.markings}
%\usepackage{fullpage}
\usepackage[left=0.75in, top=1.25in, right=0.75in, bottom=1.25in]{geometry}
\newcommand{\lambdabar}{{\mkern0.75mu\mathchar '26\mkern -9.75mu\lambda}}
\newcommand{\Jmath}{J}

\numberwithin{equation}{section}
\numberwithin{figure}{section}

\begin{document}

\title{An Introduction to Cosmic Rays}
\author{Ellen Zweibel}
\date{July 26, 2016}

\maketitle

\section{Lecture 1}

Why should we study cosmic rays? There are two parts of motivations. On the
collective side we are concerned about the totality of cosmic rays: we want to
understand better how energy is partitioned between thermal gas, magnetic
fields, and cosmic rays in the ISM/ICM. We want to understand how cosmic rays
interact with thermal gas. On the particle side we want understand how cosmic
rays are produced and the energy spectrum.

The plan of the lecture follows this bifurcation. We will talk about he particle
picture and the fluid picture. Then we will talk about diffusion which will lead
to a self-consistent diffusion model with classical cosmic ray hydrodynamics.
There is also the generalized cosmic ray hydrodynamics. Both will lead to the
discussion of applications to galactic winds.

Some early history of cosmic ray astrophysics. In 1912 Hess showed the source of
atmosphere ionization are cosmic. The ionization increases with height, which is
an evidence that the source is not terrestrial. In 1927 Clay showed ionizing
flux is latitude dependent, suggesting ``rays'' are charged particles. In 1934
Baad and Zwicky proposed that cosmic rays are from supernovae. In 1949 Hall and
Hiltner observed galactic magnetic field, and in the same year Fermi proposed the
acceleration mechanism.

One way to detect cosmic rays is through Cherenkov detectors in large water
banks. The energy spectrum is a broken power law
\begin{equation}
  \label{eq:1}
  \begin{split}
    N(E) &\sim E^{-2.7}, \quad E_\mathrm{PeV} < 3 \\
    &\sim E^{-3.0}, \quad 3 < E_\mathrm{PeV} < 100
  \end{split}
\end{equation}
The energy density is about $1\,\mathrm{eV/cm^{3}}$ which is in equipartition
with the magnetic and thermal turbulent energy density of interstellar gas.
Most of the pressure comes from $\sim \mathrm{GeV}$ particles.

The composition of cosmic rays is similar to solar system composition, but they
are rich in Li, Be, B. This can be explained by collisions of C, N, O cosmic
rays with interstellar gas which shatters them to smaller atoms. Cosmic rays are
also not enriched in r-process elements which suggests they are not direct
ejecta of SNe.

We can use $\tensor[^{10}]{\mathrm{Be}}{}$ to date the confinement time of
cosmic rays because it is an unstable isotope with half life of about a million
years.

One can't talk about cosmic rays without talking about interstellar magnetic
field. We compute the galactic magnetic field by measuring the Faraday
rotation of the ($\sim 38000$) extragalactic sources. There is a coherent and
nearly azmuthal component of the $B$ field nearly tangent to the galactic plane.
The total field is about $\sim 5\,\mathrm{\mu G}$.

If we separate the spectrum element by element, it becomes really complicated,
and every element has different features. This probably represents some joint
effect of acceleration and propagation.

The distribution of cosmic rays arriving on earth is highly isotropic to better
than $0.1\%$, and it becomes more anisotropic at energies higher than about
$10^{16}\,\mathrm{eV}$.

Remote sensor data shows that dense molecular clouds are sources of
$\gamma$-ray emission. One can get more quantitative by fitting the $\gamma$-ray
spectrum and radio spectrum of other galaxies. By fitting the radio and
$\gamma$-ray spectra one can test the assumption of equipartition at various
different galaxies. Equipartition holds in the milky way, but it can't keep up
in galaxies with higher and higher star formation rates.

There is a tight correlation between far-infrared radio luminosity and
synchrotron luminosity, which seems to hold at least to $z \sim 2$.

Lets draw some inferences from the above. From light element abundances we can
infer about $2.6\times 10^7\,\mathrm{yr}$ galactic confinement time for cosmic
rays. Similar processes occur in other galaxies. Source spectrum is about
$E^{-(2.0-2.2)}$, and source power is equivalent to about $10\%$ of SN energy
input.

What are we going to do with all these? From particle point of view we want to
explain the mechanism of acceleration and understand how propagation affects
them. From the collective view we want to understand how they modify the
structure and energetics of the ambient medium.

If we look at the particle orbits in galactic magnetic field, one can find that
Larmor radius is $r_L\sim 10^{12}\,\mathrm{cm}$. We can't really put these
cosmic ray particles into our simulations because the scale is too tiny. We need
to develop a statistical picture for transport and diffusion in phase space, and
this will help us developing a fluid picture for global feedback.

The elements of field-particle interaction include many topics, including
gyromotion, drifts, etc. For example grad-B drift will move particles along the
$B$ field lines, and it is most effective when the $B$ field varies at the same
length scale as the particle gyro radius. There is also magnetic mirroring,
where we have an adiabatic invariant $\mu = p_{\perp}^2/B$. There is
gyroresonant scattering. Since particle orbits follow field lines and short
wavelength fluctuations of the field lines average out. However if the
Doppler shifted fluctuation frequency is similar to the gyrofrequency of the
particle, its motion will grow in amplitude. Finally there is Landau resonance,
which is the condition where $\omega = k_{\parallel}v_{\parallel}$. Resonant
particles can exchange energy with a wave through parallel electric field, and
this is the mechanism for Landau damping. Wave dissipation by this mechanism is
sometimes also called transit damping.

An example of cross-field transport is (Desiati \& EZ 2014), or
(arXiv:1402.1475).

Can we quantify this transport? We define the running diffusion tensor
\begin{equation}
  \label{eq:2}
  D_{ij}(t) = \frac{1}{2N}\sum_{n=1}^N\frac{\left[ x_{i,n}(t) - x_{i,n}(0) \right][x_{j,n}(t) - x_{j,n}(0)]}{\Delta t}
\end{equation}
the diffusivity in the parallel and perpendicular directions are quite
different:
\begin{equation}
  \label{eq:3}
  \kappa_{\parallel} = \left\langle \frac{v_{parallel}^2}{\nu} \right\rangle = \frac{v^2}{3\nu} ,\quad \kappa_{\perp} = \kappa_{\parallel} = \frac{r_g^2\nu}{3}
\end{equation}

I fell asleep\dots Details plz refer to the paper.

Next we will talk about cosmic ray spectrum and 2nd order Fermi acceleration,
derive a Fokker-Planck equation and a transport equation.

\subsection{Analytics}

Lets first talk about Fermi's argument. Suppose particles are injected into our
system at a constant rate and they can be lost statistically. Lets denote the
number of cosmic ray particles that enter our system in a time period $dt$ by
$\tilde{n}(t)dt = \tilde{n}_0e^{-t/\tau_L}dt$, where $\tau_L$ is a time scale of
particle loss. If we know how much energy is inserted at different times, we can
invert that and define
\begin{equation}
  \label{eq:4}
  n(E)dE = \tilde{n}(t(E))dt,\quad n(E) = \tilde{n}(t(E))\frac{dt}{dE}
\end{equation}

Lets suppose that $dE/dt = E/\tau_\mathrm{accel}$, which is some acceleration
law. Then this equation is very easy to solve
\begin{equation}
  \label{eq:5}
  E = E_0e^{t/\tau_\mathrm{accel}},\quad t(E) = \tau_\mathrm{accel}\ln \left( \frac{E}{E_0} \right)
\end{equation}
If we implement this into our distribution equation we get
\begin{equation}
  \label{eq:6}
  n(E) = \frac{\tilde{n}_0\tau_\mathrm{accel}}{E_0}\left( \frac{E}{E_0} \right)^{-(1 + \tau_\mathrm{accel}/\tau_L)}
\end{equation}
This is a power law spectrum, and the spectrum steepness is related to the
relation between acceleration time scale and loss time scale.

In this model the oldest cosmic rays are the most energetic, however in our data
we saw that energetic cosmic rays have less confinement time, so they are
younger. This means data is against this kind of distributive model.

To make this kind of spectrum work with observation we need $\tau_\mathrm{accel}
\sim \tau_L$. There are so many ways to accelerate particles and so many
ways to lose particles, how do we make sure this is true?

Fermi considered the so-called ``magnetic clouds''. These clouds move back and
forth at random at velocity $v$, and particles can be trapped in them, and can
be reflected off them. When particles are reflected at a moving mirror they can
either gain or lose energy. Consider a head-on collision, the particle with
momentum $p$ colliding a cloud with velocity $v$, then $\Delta E = +2pv$,
however if particles are overtaking the cloud, then $\Delta E = -2pv$. If the
motions are all random then there is no net acceleration. However if we denote
the average separation of the clouds as $L$, then the frequency of head-on
collisions will be
\begin{equation}
  \label{eq:7}
  \nu = \frac{c + v}{L}
\end{equation}
whereas those overtaking collisions will be $\nu = (c - v)/L$. Therefore if we
sum over many collisions the net gain of energy will be
\begin{equation}
  \label{eq:8}
  \frac{dE}{dt} = 2pv \left( \frac{c + v}{L} - \frac{c - v}{L} \right) = \frac{4pv^2}{L} = \frac{4Ev^2}{c^2L}
\end{equation}
This has exactly the same form as the one assumed above, which leads to a power
law spectrum. The acceleration time scale is about $\tau_\mathrm{accel} \sim
c^2L/v^2c$. What is this number? In galaxies $c^2/v^2 \sim 10^9$, the separation
between the clouds is around $L\sim 3\times 10^{20}\,\mathrm{cm}$. Putting
everything together the acceleration time would be $\tau \sim
10^{19}\,\mathrm{s} \sim 3\times 10^{11}\,\mathrm{yr}$. We also know that the
loss time is about $\tau_L\sim 2.6\times 10^7\,\mathrm{yr}$. This gives
extremely steep spectrum which does not work at all. This is called second order
Fermi acceleration because $v^2$ shows up in the time scale.

Lets consider diffusion. We define the pitch angle of scattering
\begin{equation}
  \label{eq:9}
  \mu = \frac{\mathbf{p}\cdot \mathbf{B}}{pB}
\end{equation}
We are going to derive a diffusion in $\mu$ space, and show that it leads to
diffusion in $x$ space and apply it to shock acceleration and worry about how to
make it self-consistent.

We are going to derive a probability function
\begin{equation}
  \label{eq:10}
  P(\mu, \Delta \mu) = \text{Probability that a particle with $\mu$ is scattered by $\Delta \mu$ in time $\Delta t$}
\end{equation}
And we normalize this function such that integrated over $\Delta \mu$ it gives
some number $\xi$. Integrating this we have a distribution function $f(\mu, t)$
which is our cosmic ray distribution function (which also depends on $x$ and
$t$). We have
\begin{equation}
  \label{eq:11}
  f(\mu, t + \Delta t) = \int f(\mu - \Delta \mu, t)P(\mu - \Delta \mu, \Delta \mu) d\Delta \mu
\end{equation}
which simply means that particles at $t + \Delta t$ with $\mu$ can come from
various ``kicks'' of different $\Delta \mu$. We are going to expand both sides,
next time.



\end{document}
